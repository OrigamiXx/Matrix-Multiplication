%%% Slides without notes
\documentclass[t,10pt,
%handout,
mathserif,xcolor=dvipsnames]{beamer}
\renewcommand{\note}[1]{}

%%% Slides with notes
%% \documentclass[t,11pt,mathserif,xcolor=dvipsnames]{beamer}
%% \usepackage{pgfpages}
%% \setbeameroption{show notes}
%% \setbeameroption{show notes on second screen=right}

%%% Handout with notes
%% \documentclass[t,11pt,handout,mathserif,xcolor=dvipsnames]{beamer} % for printing
%% \usepackage{pgfpages}
%% \setbeameroption{show notes}
%% \setbeameroption{show notes on second screen=right}

%%% Handout without notes
%% \documentclass[t,11pt,handout,mathserif,xcolor=dvipsnames]{beamer}
%% \renewcommand{\note}[1]{}

% includes
\usepackage{beamerthemeMWA}

\usepackage{verbatim}
\usepackage[T1]{fontenc}
\usepackage{ae,aecompl}
\usepackage{amsmath}
\usepackage{soul}
\usepackage{graphicx}
\usepackage{latexsym}
\usepackage{amssymb}
\usepackage{xspace}
\usepackage{amsbsy}
\usepackage{amsthm}
\usepackage{listings}
\usepackage{enumerate}
\usepackage{pgf}
\usepackage{tikz}
\usepgflibrary{arrows}


\lstset{basicstyle=\ttfamily\scriptsize, % print whole listing small
stringstyle=\ttfamily,
keywordstyle=\color{black}\underbar,
sensitive=true,
%identifierstyle=\color{cerulean},
showstringspaces=false}

\renewcommand{\qedsymbol}{\ensuremath{_\blacksquare}}

\setcounter{tocdepth}{4}

\setbeamertemplate{navigation symbols}{}

\usepackage[absolute,overlay]{textpos} 
%\usepackage[colorgrid,texcoord]{eso-pic} 

%adjust the TPHorizModule and TPHorizModule units to the displayed mm %grid 
\TPGrid{210}{297} 



%\usepackage{a4wide}
\usepackage[utf8]{inputenc}
\usepackage{amssymb}
\usepackage{amsmath}
\usepackage{amsthm}
%\usepackage[amsmath,hyperref,thmmarks]{ntheorem}
\usepackage{xspace}
\usepackage{ulem}
\usepackage{graphicx}
\graphicspath{figs}
\usepackage{mdwlist}
\usepackage{color}
\usepackage{soul}
\usepackage{authblk}
\usepackage{algpseudocode}
\usepackage{xparse}
\usepackage{tikz}
%\usepackage{3dplot}
%\usepackage[active,tightpage]{preview}  %generates a tightly fitting border around the work
%\PreviewEnvironment{tikzpicture}
%\setlength\PreviewBorder{2mm}
%% \usetikzlibrary{external}
%% \tikzexternalize
%% \tikzexternaldisable
%% \tikzset{external/force remake=true}
%% \tikzset{external/system call={%
%%   pdflatex \tikzexternalcheckshellescape -halt-on-error -interaction=batchmode -jobname "\image" "\texsource"; 
%%   ps2pdf13 "\image".pdf "\image-13".pdf && cp "\image-13".pdf "\image".pdf}}



\usepackage{changebar}
\setcounter{changebargrey}{0} % darkness of change bars (0 = black)


% Put a period after a paragraph title.
%\let\originalparagraph\paragraph
%\renewcommand{\paragraph}[2][.]{\originalparagraph{#2#1}}


% Margin notes
\newcommand\marginnote[1]{\marginpar{\raggedright\footnotesize\emph{#1}}}

% Symbol to end definitions.
\newcommand{\defnend}{\hfill\ensuremath{\blacktriangleleft}}

% -----------------------------------------------------------------------------
% Theorem environments
% -----------------------------------------------------------------------------

\newtheorem{thm}{Theorem}
\newtheorem{lem}[thm]{Lemma}
\newtheorem{question}[thm]{Question}
\newtheorem{conj}[thm]{Conjecture}
\newtheorem{claim}[thm]{Claim}
\newtheorem{cor}[thm]{Corollary}
\newtheorem{obs}[thm]{Observation}
\newtheorem{prop}[thm]{Proposition}


%% \newtheorem{thm}{Theorem}
%% \newtheorem{lem}[thm]{Lemma}
%% \newtheorem{question}[thm]{Question}
%% \newtheorem{conj}[thm]{Conjecture}
%% \newtheorem{fact}[thm]{Fact}
%% \newtheorem{claim}[thm]{Claim}
%% \newtheorem{cor}[thm]{Corollary}
%% \newtheorem{definition}[thm]{Definition}
%% \newtheorem{remark}[thm]{Remark}
%% \newtheorem{prop}[thm]{Proposition}
%% \newtheorem{obs}[thm]{Observation}

\newcommand{\mathcolor}{\color{ta2gray}}
\newcommand{\inl}[1]{$\mathcolor #1$}
\newcommand{\stress}[1]{\emph{\structure{#1}}}
\newcommand{\cit}[1]{\textbf{\structure{\small  #1}}}
\newcommand{\IFPC}{\logic{FPC}}
\newcommand{\titl}[1]{\textbf{\structure{\small  #1}}}
\newcommand{\displaym}[1]{$$\mathcolor #1 $$}
\newcommand{\DLT}{\ensuremath{\mathrm{DLogTime}}\xspace}
\newcommand{\Part}{\mathcal{P}}
\newcommand{\E}{\ensuremath{\mathcal{E}}}
\newcommand{\orb}{\mathrm{Orb}}
\newcommand{\Aa}{\ensuremath{\mathbb{A}}}
\newcommand{\ra}{\rightarrow}


\newcommand{\F}{\ensuremath{\mathbb{F}}}
\newcommand{\de}{:=}
\newcommand{\set}[1]{\ensuremath{\{{#1}\}}}
\newcommand{\condset}[2]{\ensuremath{\set{{#1}\;|\;{#2}}}}
\newcommand{\xs}{\ensuremath{x_1,\ldots,x_n}}
\newcommand{\G}{\ensuremath{\mathcal{G}}\xspace}
\newcommand{\X}{\ensuremath{\mathcal{X}}\xspace}
\newcommand{\p}{\partial}
\newcommand{\V}[1]{\ensuremath{\mathrm{var}({#1})}}
\newcommand{\rk}{\mathrm{rank}}
\newcommand{\D}{\mathrm{Det}}
\newcommand{\nev}{\not\equiv}
\newcommand{\NN}{\ensuremath{\mathbb{N}}\xspace}
\newcommand{\bs}{\ensuremath{\backslash}}
%\newcommand{\C}{\ensuremath{\mathcal{C}}\xspace}
\newcommand{\M}{\ensuremath{\mathcal{M}}\xspace}
\newcommand{\U}{\ensuremath{\mathcal{U}}\xspace}
\newcommand{\Op}{\ensuremath{\mathcal{O}}\xspace}
\newcommand{\s}{\ensuremath{\mathcal{S}}\xspace}
\newcommand{\ev}{\equiv}
\newcommand{\la}{\leftarrow}
\newcommand{\B}{\mathbb{B}}
\newcommand{\C}{\mathbb{C}}
\newcommand{\A}{\mathcal{A}}



\newcommand{\Sym}[1]{\ensuremath{\mathrm{Sym}_{#1}}\xspace}
\newcommand{\Stab}{\ensuremath{\mathrm{Stab}}\xspace}

\newcommand{\poly}{\ensuremath{\mathrm{poly}}\xspace}

\newcommand{\fin}{\mathrm{fin}}
\newcommand{\logic}[1]{\text{\upshape #1}\xspace}
\newcommand{\dom}{\ensuremath{\mathrm{dom}}}
\newcommand{\AC}[1]{\ensuremath{\mathrm{AC}^{#1}}\xspace}
\newcommand{\NC}[1]{\ensuremath{\mathrm{NC}^{#1}}\xspace}
\newcommand{\FO}{\logic{FO}}
\newcommand{\FOARB}{\ensuremath{(\FO+\mathrm{arb})}\xspace}
\newcommand{\FOA}{\ensuremath{(\FO\mathrm{+arith)}}\xspace}
\newcommand{\FOa}[1]{\ensuremath{\FO(#1)}\xspace}
\newcommand{\CPTC}{\logic{$\tilde{\text C}$PT(Card)}}
\newcommand{\NUM}{\ensuremath{\mathrm{NUMBER}}}
\newcommand{\FOC}{\ensuremath{\FO\mathrm{+C}}\xspace}
\newcommand{\PT}{\ensuremath{\logic{P}}\xspace}
\newcommand{\Pp}{\ensuremath{\PT/\poly}\xspace}
\newcommand{\BPP}{\ensuremath{\mathrm{BPP}}\xspace}
\newcommand{\NP}{\ensuremath{\logic{NP}}\xspace}
\newcommand{\LFP}{\logic{LFP}}
\newcommand{\LFPC}{\logic{LFPC}}
\newcommand{\FP}{\logic{FP}}
\newcommand{\FPR}{\logic{FPR}}
\newcommand{\FF}[1]{\mathbb{F}_{#1}}

\newcommand{\FPC}{\logic{FPC}}
\newcommand{\Orb}[2]{\ensuremath{\mathrm{Orb}_{#1}({#2})}\xspace}
\newcommand{\OP}[2]{\ensuremath{{#1}\text{-OP}_{#2}\xspace}}
\renewcommand{\implies}{\Rightarrow}
\newcommand{\qd}{\quad}
\newcommand{\SP}{\mathrm{Sup}}
\newcommand{\spt}{\mathrm{spt}}
\newcommand{\sse}{\subseteq}
\newcommand{\ssn}{\subsetneq}
\newcommand{\spe}{\supseteq}
\newcommand{\spn}{\supsetneq}
\newcommand{\es}{\emptyset}
\newcommand{\inv}{^{-1}}
%\newcommand{\so}{{\ensuremath{\langle S\rangle}\xspace}}
%\newcommand{\ls}{\le_\so}
\newcommand{\Perm}[1]{\ensuremath{\mathrm{Per}({#1})}\xspace}

\newcommand{\com}[1]{\noindent\fbox{\hl{#1}}}
\renewcommand{\bar}[1]{\overline{#1}}

\newcommand{\LT}[1]{\ensuremath{\mathrm{LT}({#1})}\xspace}
\newcommand{\LM}[1]{\ensuremath{\mathrm{LM}({#1})}\xspace}

\newcommand{\FOt}{\ensuremath{\FO(\tau)}\xspace}
\newcommand{\LFPCt}{\ensuremath{(\LFPC)(\tau)}\xspace}

\newcommand{\hdef}[2]{\structure{#1}\; #2}

\newcommand{\alignedeq}[1]{
\begin{equation*}
\begin{aligned}
#1
\end{aligned}
\end{equation*}
}

\newcommand{\deffunc}[3]{
\begin{flalign*}
{#1} :&~#2 \\
    &~#3
\end{flalign*}
}

\newcommand{\deffunca}[3]{
{#1} :&~#2 \\
    &~#3 \\
}

\newcommand{\alg}[2]{
\hrule\smallskip
\noindent{#1}
\smallskip
\hrule
\vspace{-1.5ex}
\begin{enumerate*}
{#2}
\end{enumerate*}
\vspace{-1.5ex}
\hrule
}

\newcommand{\algio}[4]{
  \alg{
    {#1}
    \smallskip
    \hrule
    \smallskip
    \hspace*{2ex}Input: {#2} \\
    \hspace*{2ex}Output: {#3} 
  }{
    {#4}
  }
}


%%%%%%% Frame template %%%%%%%%%%

%% \begin{myframe}{ }
%%   \begin{itemize}
%%   \item
%%   \end{itemize}
%% \end{myframe}

%%%%%%% End Frame template %%%%%

%\setbeamertemplate{footline}[page number]

\tikzset{
  every overlay node/.style={
    draw=white,fill=white,rounded corners,anchor=north west,
  },
}
% Usage:
% \tikzoverlay at (-1cm,-5cm) {content};
% or
% \tikzoverlay[text width=5cm] at (-1cm,-5cm) {content};
\def\tikzoverlay{%
   \tikz[baseline,overlay]\node[every overlay node]
}%


\newcommand{\QQ}{\ensuremath{\mathbb{Q}}}

\title{\textbf{Computer-Aided Search for \\ Matrix Multiplication Algorithms}}

\institute{{\large \structure{Matthew Anderson} \quad  Zongliang Ji \quad  Anthony Yang Xu \\ \includegraphics[width=1.3in]{figs/uc_logo.eps}}}
\date{December 13, 2017 \\[1ex] {\small Simons Institute for the Theory of Computing}}

\begin{document}
\begin{frame}[plain]
  
  \vspace{5ex}

  \titlepage

\end{frame}

\section{Introduction}

\subsection{The Matrix Multiplication Problem}

\begin{myframe}{Matrix Multiplication}

  \begin{problem}%[Square Matrix Multiplication]
    \textbf{Input:} $A \in \F^{n \times n}$, $B \in \F^{n \times n}$ \\[1ex]
    \textbf{Output:} $C = A \times B \in \F^{n\times n}$.    
  \end{problem}

  For example:
  $$
  \left[
    \begin{tabular}{cc}
      1 & 2 \\
        2 & 0
    \end{tabular}
    \right]
  \times
  \left[
    \begin{tabular}{cc}
      -1 & 3 \\
      1 & 1
    \end{tabular}
    \right]
  =
  \left[
    \begin{tabular}{cc}
      1 & 5 \\
      -2 & 6
    \end{tabular}
    \right]    
  $$
  
  
  %\item Focus on \structure{square} matrix multiplication where $m = n = p$.
  How many operations does it take to multiply two $n$-by-$n$
  matrices?
  \begin{itemize}
  \item $O(n^3)$ by naively computing $n^2$ dot products of rows of $A$ and columns of $B$.
  \item $\Omega(n^2)$ because there are at $n^2$ cells to output.
  \end{itemize}
  \begin{question}
  What is the smallest $\omega \le 3$ such that $n$-by-$n$ matrix
  multiplication can be done in time $O(n^\omega)$?
  \end{question}
    
\end{myframe}

\subsection{Historical Context}

\begin{myframe}{Progress on $\omega$}

  \begin{center}
  {\Large
  \begin{tabular}{ll}
    \structure{$3$} & Naive \\
    \structure{$\uline{2}.808$} & Strassen 1969 \\
    \structure{$2.\uline{7}96$} & Pan 1978 \\
    \structure{$2.7\uline{8}$} & Bini et al 1979 \\
    \structure{$2.\uline{5}22$} & Schönhage 1981 \\
    \structure{$2.\uline{4}96$} & Coppersmith \& Winograd 1982 \\
    \structure{$2.4\uline{7}9$} & Strassen 1986 \\  % Laser
    \structure{$2.\uline{3}75477$} & Coppersmith \& Winograd 1987 \\
    \structure{$2.37\uline{4}$} & Stothers 2010 \\
    \structure{$2.37\uline{2}8642$} & Williams 2011 \\
    \structure{$2.37286\uline{3}9$} & Le Gall 2014
  \end{tabular}
  }
  \end{center}
  
\end{myframe}

\begin{frame}[label=outline]{\vspace{1mm}\textbf{Outline}}

  \begin{itemize}
  \item Introduction
  \item Cohn-Umans Framework
  \item Checking
  \item Search
  \item Lessons
  \end{itemize}
  
\end{frame}

\section{Cohn-Umans Framework}

\newcommand\FFT{\mathrm{FFT}}
\newcommand\CC{\mathbb{C}}

\begin{myframe}{Cohn-Umans Framework}

  In 2003, Cohn and Umans proposed an approach for improving the upper
  bound on $\omega$.

  \begin{itemize}
  \item Inspired by the $\Theta(n \log n)$ FFT-based algorithm for
    multiplying two degree $n$ univariate polynomial, c.f., e.g.,
    [CLRS 2009, Chap 30].

    $$ A \times B = C\text{ becomes }  \FFT^{-1}(\FFT(A) * \FFT(B)) = C$$

  \end{itemize}

  \pause
  
  \structure{Idea} determine a suitable group $G$ to embed
  multiplication into the group algebra $\CC[G]$ using sets $X, Y, Z
  \sse G$, with $|X| = |Y| = |Z| = n$.

  $$\bar{A} = \sum_{i, j \in [n]} (x_i^{-1} y_j) A_{i,j}, \quad \bar{B} =
  \sum_{j, k \in [n]} (y_j^{-1} z_k) B_{j,k}, \quad \bar{C} = \sum_{i, k \in
    [n]} (x_i^{-1} z_k) C_{i,k}$$

  where \structure{triple product property} holds: $\forall x,x'
  \in X, \forall y,y' \in Y, \forall z,z' \in Z,$

  $$x^{-1} y y'^{-1} z = x'^{-1} z' \text{ iff } x = x', y = y', z =
  z'.$$
  
  $\omega$ implied by $G$ depends on $|G|$ and aspects of its
  representation.
  
\end{myframe}



\subsection{Strong Uniquely Solvable Puzzles}

\begin{myframe}{Puzzles}

  \begin{definition}[Puzzle]
    An $(s,k)$-\emph{\structure{puzzle}} is a subset $P \sse U_k = \set{1,2,3}^k$ with $|P| = s$.
  \end{definition}

  \medskip
  
  Consider
  \begin{equation*}
    \begin{aligned}
      P = \{&(3,1,3,2), (1,2,3,2), (1,1,1,3),\\ &(3,2,1,3),(3,3,2,3)\} \hspace{50ex}
    \end{aligned}
  \end{equation*}
  
  \begin{itemize}
  \item $P$ is a (5,4)-puzzle.
  \item $P$ has five \emph{rows}.
  \item $P$ has four \emph{columns}.
  \end{itemize}

  \tikzoverlay[text width=3cm] at (7.5cm,3.5cm) {
    $P$\\[.5ex]
    \begin{tabular}{|c|c|c|c|}
      \hline
      3&1&3&2 \\ \hline
      1&2&3&2 \\ \hline
      1&1&1&3 \\ \hline 
      3&2&1&3 \\ \hline 
      3&3&2&3 \\ \hline
    \end{tabular}
  };
  
  Note that:
  \begin{itemize}
  \item The columns are ordered.
  \item The rows are unordered (as $P$ is a set).
  %\item It sometimes helps to consider elements of $U_k$ as ordered lexicographically.
  \end{itemize}

  
\end{myframe}

\begin{comment}
\begin{myframe}{\uline{Uniquely Solvable} Puzzles -- Intuition}

  % XXX - cut this slide and just say this on the next slide?  Include
  % the last remark on the following slide.
  
  It helps to think of each row of a puzzle as consisting of
  \structure{three} pieces corresponding to the coordinates of the row
  that are 1, 2, or 3.


  \begin{itemize}
  \item E.g., the row $(1, 2, 1, 1)$ consists of a 1-piece $(1, *, 1,
    1)$, a 2-piece $(*, 2, *, *)$ and a 3-piece $(*, *, *, *)$.
    
  \item A 1-piece, 2-piece, and 3-piece can be fit together to form a
    row iff there is exactly one non-$*$ for each coordinate among the
    three pieces.
  
  \item E.g., pieces $(1, 1, *, *)$, $(*, *, 2, *)$, $(*, 3, 3, *)$
    cannot be fix together because they overlap on the $2^{nd}$ and
    $3^{rd}$ coords, and there is no entry on the $4^{th}$.

  \end{itemize}

  Informally, a puzzle $P$ is \emph{\structure{uniquely solvable}} if
  there is no way to reorganize the 1-, 2-, 3-pieces of $P$ into a
  puzzle different from $P$.

  \pause
  
  \begin{itemize}
  \item This is a natural property that holds of ``good'' real-world puzzles:
    \begin{itemize}
    \item jigsaw puzzles (locally), and
    \item sudoku puzzles (globally).
    \end{itemize}
  \end{itemize}
  
\end{myframe}
\end{comment}


\begin{myframe}{Uniquely Solvable Puzzles -- Intuition}

  We're interested in puzzles that are \structure{uniquely solvable}.
  
  \medskip
  
  \includegraphics<2>[width=\linewidth]{figs/usp1.eps}
  \includegraphics<3>[width=\linewidth]{figs/usp2.eps}
  \includegraphics<4>[width=\linewidth]{figs/usp3.eps}
  \includegraphics<5>[width=\linewidth]{figs/usp4.eps}
  \includegraphics<6>[width=\linewidth]{figs/usp5.eps}
  \includegraphics<7>[width=\linewidth]{figs/usp6.eps}

  \smallskip
  
  \begin{itemize}
  \item<2->This puzzle is not uniquely solvable. \smallskip
  \item<5->Can be witnessed by three permutations: \\
    $\pi_1 = (1)(2)(3)(4)(5)$\\
    $\pi_2 = (1)(2\;3\;5)(4)$\\
    $\pi_3 = (1)(2\;5\;3)(4)$\\ \smallskip
  \item<7->Since the resulting puzzles is not the same as the original
    puzzle (even reordering rows), the puzzle is not \structure{uniquely solvable}.
  \end{itemize}
  
    
\end{myframe}


\begin{myframe}{Uniquely Solvable Puzzles -- Formal}
  
  \begin{definition}[Uniquely Solvable Puzzle]
    An $(s,k)$-puzzle $P$ is \emph{\structure{uniquely solvable}} if
    $\forall \pi_1, \pi_2, \pi_3 \in \Sym{P}:$
    \begin{enumerate}
    \item either $\pi_1 = \pi_2 = \pi_3$, or
    \item $\exists r \in P, \exists i \in [k]$ such that \structure{at least} two
      of the following hold:
      \begin{enumerate}
      \item $(\pi_1(r))_i = 1$,
      \item $(\pi_2(r))_i = 2$,
      \item $(\pi_3(r))_i = 3$.
      \end{enumerate}
    \end{enumerate}
    
  \end{definition}

  Basically, for every way of non-trivial way of reordering the 1-,
  2-, 3-pieces according to $\pi_1, \pi_2, \pi_3$, they cannot all fit
  together.

  \pause
  
  \begin{itemize}
  \item This is a natural property that holds of ``good'' real-world puzzles:
    \begin{itemize}
    \item jigsaw puzzles (locally), and
    \item sudoku puzzles (globally).
    \end{itemize}
  \end{itemize}
  
  
\end{myframe}

\begin{myframe}{\uline{Strong} Uniquely Solvable Puzzles}

  \begin{definition}[Strong Uniquely Solvable Puzzle]

    An $(s,k)$-puzzle $P$ is \emph{\structure{strong uniquely solvable}} if
    $\forall \pi_1, \pi_2, \pi_3 \in \Sym{P}:$
    \begin{enumerate}
    \item either $\pi_1 = \pi_2 = \pi_3$, or
    \item $\exists r \in P, \exists i \in [k]$ such that \structure{exactly} two
      of the following hold:
      \begin{enumerate}
      \item $(\pi_1(r))_i = 1$,
      \item $(\pi_2(r))_i = 2$,
      \item $(\pi_3(r))_i = 3$.
      \end{enumerate}
    \end{enumerate}

  \end{definition}

  \pause
  
  No good intuition for the ``exactly two'' part, but a useful
  implication.

  \begin{lemma}[{[CKSU 05, Corollary 3.6]}]
    For an integer $m \ge 3$, if there is a strong uniquely solvable
    $(s,k)$-puzzle, 

    $$\omega \le \frac{3 \log m}{\log(m-1)} - \frac{3 \log s!}{sk \log(m-1)}.$$
  \end{lemma}
  
\end{myframe}

\begin{myframe}{Useful Strong Uniquely Solvable Puzzles}

  \begin{lemma}[{[CKSU 05, Proposition 3.8]}]
   There is an infinite family of SUSP that achieve $\omega < 2.48$.
  \end{lemma}

  \medskip
  There are group-theoretic constructions derived from [Strassen 86]
  and [Coppersmith-Winograd 87] that achieve the $\omega$'s of those
  works.

  \bigskip

  \pause
  
  \begin{lemma}[{[BCCGU 16]}]
  SUSP cannot show $\omega < 2 + \epsilon$, for some $\epsilon >
  0$.
  \end{lemma}
  \begin{itemize}
  \item This was conditionally true if the Erd\"{o}-Szemeredi
    sunflower conjecture held [Alon-Shpilka-Umans 2013].
  \item Recent progress on cap sets and arithmetic progressions made
    this unconditional [Ellenberg 2016, Croot-Lev-Pach, 2016].
  \end{itemize}

  
\end{myframe}


\begin{myframe}{Our Goals \& Approach}

  \structure{Goal} Find strong uniquely solvable puzzles (SUSP) that imply smaller $\omega$.

  \medskip

  \pause
  
  \structure{Approach} 
  \begin{itemize}
  \item For fixed width $k$, the larger height $s$ of a SUSP is, the
    smaller $\omega$ is implied. We want to determine for small values
    of $k$, the maximum $s$ that can be achieved.  Hopefully, this
    leads to an improvement in $\omega$.
  \item Develop software platform to explore and experiment with SUSP.
  \item \structure{Algorithm Design}
    \begin{itemize}
    \item Checking that a puzzle is a SUSP.
    \item Searching for large SUSP.
    \end{itemize}
  \item \structure{Implementation}
    \begin{itemize}
    \item Targeted mainly desktop but also HPC environments.
    \end{itemize}
  \item We only need to find one sufficiently large puzzle to achieve
    a new algorithm -- worst-case performance isn't a good metric!
  \end{itemize}

  \medskip
  \pause
  
  \structure{Secondary Goal} Develop a theory research program that
  undergraduates can meaningfully participate in.
  
  
  
\end{myframe}

\section{Checking}

\againframe{outline}

\newcommand\coNP{\ensuremath{\mathrm{coNP}}}

\begin{myframe}{Checking}

  \begin{problem}[SUSP-Check]
    \textbf{Input:} A $(s,k)$-puzzle $P$. \\[1ex]
    \textbf{Output:} True iff $P$ is a strong uniquely solvable puzzle.
  \end{problem}

  It suffices to evaluate the following formula for a puzzle $P$:\\[-2ex]
  \begin{equation*}
    \begin{aligned}
      \forall \pi_1&,\pi_2, \pi_3 \in \Sym{P}.\\ &~~~\pi_1 = \pi_2 = \pi_3 \\ &\vee \exists r \in P. \exists i \in [k]. ((\pi_1(r))_i = 1) + ((\pi_2(r))_i = 2) + ((\pi_3(r))_i = 3) = 2
    \end{aligned}
  \end{equation*}
  
  \begin{itemize}
  \item That a $P$ is not a SUSP is be witnessed by permutations
    $\pi_1, \pi_2, \pi_3$.
  \item SUSP-Check is in \coNP.
  \item When we only want to verify uniquely solvability it is
    reducible to graph automorphism.
  \item It is not clear whether SUSP-Check is \coNP-hard.
  \end{itemize}

  
\end{myframe}

\subsection{Brute Force}

\begin{myframe}{Brute Force}

  ~\\[-7ex]
  \begin{equation*}
    \begin{aligned}
      \forall \pi_1&,\pi_2, \pi_3 \in \Sym{P}.\\
      &~~~\pi_1 = \pi_2 = \pi_3\\
      &\vee \exists r \in P. \exists i \in [k]. ((\pi_1(r))_i = 1) + ((\pi_2(r))_i = 2) + ((\pi_3(r))_i = 3) = 2
    \end{aligned}
  \end{equation*}
  
  \begin{itemize}
  \item Brute force model checking takes $O((s!)^3\cdot\poly(s,k))$ time.
  \item Easy to implement.
  \item Run time makes it practically useless for puzzles with width $k > 4$.
  \item Served as a reference implementation for debugging.
  \item Good for getting students feet wet with relevant issues with
    implementation and mathematical objects.
  \item It will be more convenient to think about checking the complement formula.
  \end{itemize}

  ~\\[-5ex]
  \begin{equation*}
    \begin{aligned}
      \exists \pi_1&,\pi_2, \pi_3 \in \Sym{P}.\\
      &~~~\pi_1, \pi_2, \pi_3 \text{ not all equal}\\
      &\wedge \forall r \in P. \forall i \in [k]. ((\pi_1(r))_i = 1) + ((\pi_2(r))_i = 2) + ((\pi_3(r))_i = 3) \neq 2
    \end{aligned}
  \end{equation*}
  
\end{myframe}

\subsection{Pruning \& Preprocessing}

\begin{myframe}{Pruning}

  ~\\[-5ex]
  \begin{equation*}
    \begin{aligned}
      \exists \alert{\pi_1}&,\pi_2, \pi_3 \in \Sym{P}.\\
      &~~~\alert{\pi_1}, \pi_2, \pi_3 \text{ not all equal}\\
      &\wedge \forall r \in P. \forall i \in [k]. (\alert{(\pi_1(r))_i} = 1) + ((\pi_2(r))_i = 2) + ((\pi_3(r))_i = 3) \neq 2
    \end{aligned}
  \end{equation*}

  \begin{itemize}
  \item Force $\pi_1 = 1$ to get:
  \end{itemize}

  ~\\[-5ex]
  \begin{equation*}
    \begin{aligned}
      \exists \pi_2,&\pi_3 \in \Sym{P}.\\
      &~~~\alert{1}, \pi_2, \pi_3 \text{ not all equal}\\
      &\wedge \forall r \in P. \forall i \in [k]. (\alert{r_i} = 1) + ((\pi_2(r))_i = 2) + ((\pi_3(r))_i = 3) \neq 2
    \end{aligned}
  \end{equation*}
  
  \begin{itemize}
  \item Results in an equivalent formula because the rows of a puzzle
    are unordered.
  \item Removes a $s!$ factor from runtime, achieving $O((s!)^2 \cdot
    \poly(s,k))$.
  \end{itemize}


  
\end{myframe}

\begin{myframe}{Preprocessing}
  
  ~\\[-5ex]
  \begin{equation*}
    \begin{aligned}
      \exists \pi_2,&\pi_3 \in \Sym{P}.\\
      &~~~\pi_2, \pi_3 \text{ not both } 1\\
      &\wedge \forall r \in P. \alert{\forall i \in [k]. (r_i = 1) + ((\pi_2(r))_i = 2) + ((\pi_3(r))_i = 3) \neq 2}
    \end{aligned}
  \end{equation*}


  \begin{itemize}
  \item The innermost $\exists$ can be precomputed in $O(s^3k)$ time
    by creating a Boolean relation $T_P \in P \times P \times P$, where
    $$(p,q,r) \in T_P \Leftrightarrow \forall i \in [k]. (r_i = 1) + ((\pi_2(r))_i = 2) + ((\pi_3(r))_i = 3) \neq 2.$$

  \item This simplifies the formula we are checking to:
    $$\exists \pi_2,\pi_3 \in \Sym{P}. \pi_2, \pi_3 \text{ not both } 1 \wedge
    \forall r \in P. \alert{(r,\pi_2(r),\pi_3(r)) \in T_P}.$$
  \item This makes the dominant term of the running time independent
    of $k$ and is useful for the next step.
  \end{itemize}
  
\end{myframe}

\subsection{Reduction to 3D Matching}


\begin{myframe}{Reduction to 3D Matching}

  This results in the formula below which is true iff $P$ is not a SUSP.\\[-3ex]
  
  $$\exists \pi_2,\pi_3 \in \Sym{P}. \pi_2, \pi_3 \text{ not both } 1 \wedge
  \forall r \in P. (r,\pi_2(r),\pi_3(r)) \in T_P.$$

  \medskip
  
  This is an instance of a natural \NP problem.

  \medskip
  
  \begin{problem}[3D Matching]
    \textbf{Input:} A 3-hypergraph $G = \langle V, E \sse V \times V
    \times V\rangle$. \\[1ex] \textbf{Output:} True iff $\exists M
    \sse E$ with $|M| = |V|$ and $\forall e_1 \neq e_2 \in M$, for
    each coordinate $e_1$ and $e_2$ are vertex disjoint.
  \end{problem}

  \medskip
  
  We can reduce verifying $P$ is not a SUSP to 3D matching.
  \begin{itemize}
  \item Consider $G_P = \langle P, T_P\rangle$.
  \item Observe that $P$ is a not a SUSP iff $G_P$ has a 3D matching
    that isn't the identity matching, i.e., $M = \set{(r_1,r_1,r_1),
      \ldots, (r_s,r_s,r_s)}$.
  \item That $M$ isn't identity matching is necessary, but not
    interesting so we won't talk about it anymore.
  \end{itemize}
  
\end{myframe}

\subsubsection{Dynamic Programming and Bidirectional Search}


\begin{myframe}{Dynamic Programming}

  We can determine 3D matchings using dynamic programming.

  \begin{itemize}
  \item Fix some ordering of $P$: $r_1,\ldots,r_s$.
  \item Consider iteratively constructing a matching $M$ of $G_P$
    where in the $i^{th}$ step you select an edge $(r_i,*,*) \in T_P$.
  \item After the $i^{th}$ step, the remaining edges that can be
    selected are $$T_P^{X,Y} = T_P \cap (\set{r_{i+1},\ldots,r_s} \times (P - Y)
    \times (P - Z))$$ where $Y$ and $Z$ are the vertices that have
    already be selected for the second and third coordinate
    respectively and $|Y| = |Z| = i$.
  \item Call $S(i, X, Y)$ the subproblem of whether a 3D matching can be
    completed on $T_P^{X,Y}$ and $i = |X| = |Y|$.
  \item Observe that $S(i,X,Y)$ has a 3D matching iff there exists
    $(r_{i+1},p,q) \in T_P^{X,Y}$ and $S(i+1, X \cup \set{a},Y \cup
    \set{b})$ has a 3D matching.
  \end{itemize}
  This gives an $O(2^{2s} s^2)$ algorithm via dynamic programming.
  
\end{myframe}

\begin{myframe}{Practical Running Time -- Dynamic Programming}

  Average checking time versus puzzle height for 50,000 (8,*)-puzzles.
  \hspace*{-9ex}\includegraphics[width=1.28\linewidth, trim={.5in 2in 1in 1in}, clip]{figs/plotuni.pdf}
 
\end{myframe}

\begin{myframe}{Dynamic Programming + Bidirectional Search}

  Perform two searches using dynamic programming:
  \begin{itemize}
  \item The first selects edges whose first coordinates are
    $r_1,r_2,\ldots,r_{\lfloor s/2\rfloor}$.
  \item The second selects edges whose first coordinates are
    $r_s,r_{s-1},\ldots,r_{\lfloor s/2\rfloor+1}$.
  \item The searches use the other's memoization table in the last
    step.
  \end{itemize}

  \bigskip
  
  This improves performance by about a squareroot.
  \begin{itemize}
  \item The worst-case running time becomes $O(2^s s^2)$.
  \item The worst-case memory usage is $O(2^s s)$.
  \end{itemize}

  \bigskip
  
  These are the best worst-case bounds we could bounds we could devise.
  
  
\end{myframe}

\begin{myframe}{Practical Running Time -- Bidirectional}

  Average checking time versus puzzle height for 50,000 (8,*)-puzzles.
  \hspace*{-9ex}\includegraphics[width=1.28\linewidth, trim={.5in 2in 1in 1in}, clip]{figs/plotbi.pdf}
 
\end{myframe}

\subsubsection{Other Reductions}

\begin{myframe}{Other Reductions}

  We tried reducing 3D matching to CNF satisfiability.
  \begin{itemize}
  \item Reduced satisfiability instance had $2s^2$ variables and
    $O(s^3)$ clauses.
  \item Used an open-source conflict-driven clause-learning SAT solver
    MapleCOMSPS that won the general category of the 2016 SAT
    Competition.  Solver written in part by Jia Hui Liang, Vijay
    Ganesh, and Chanseok Oh.  \\\url{http://www.satcompetition.org}
  \end{itemize}

  \bigskip

  We tried reducing 3D matching to 0-1 integer programming.
  \begin{itemize}
  \item Reduced IP instance had $s^3$ variables and
    $O(s^3)$ equations.
  \item Used a close-source optimization library Gurobi.
    \url{http://www.gurobi.com}
  \end{itemize}
  
\end{myframe}

\begin{myframe}{Practical Running Time -- SAT / IP}

  Average checking time versus puzzle height for 50,000 (8,*)-puzzles.
  \hspace*{-9ex}\includegraphics[width=1.28\linewidth, trim={.5in 2in 1in 1in}, clip]{figs/plotsat.pdf}
 
\end{myframe}


\begin{myframe}{Implementation}

  Current implementation is hybrid of several algorithms.
  \begin{itemize}
  \item Brute force for very small instances, $k \le3$.
  \item Bidirectional Dynamic programming for moderate instances $k
    \le 6$.
  \item SAT for large instances with $k > 6$ and $s < 40$.
  \item IP for all bigger instances.
  \end{itemize}

  \bigskip

  We implemented a number of heuristics that are not always conclusive,
  but frequently can determine the result early.
  \begin{itemize}
  \item Briefly trying to randomly or greedily generate 3D matchings.
  \item Verifying that all pairs of rows or triples of rows form a SUSP.
  \item Testing whether the puzzle is uniquely solvable using the graph
    isomorphism library Nauty: \url{http://users.cecs.anu.edu.au/~bdm/nauty/}
  \item Simplifying the 3D matching instance using properties of the
    puzzle, e.g., using that a column only contains two of $\set{1,2,3}$.
  \end{itemize}
  
\end{myframe}


\subsection{Results}

\begin{myframe}{Practical Running Time -- Final}

  Average checking time versus puzzle height for 50,000 (8,*)-puzzles.
  \hspace*{-9ex}\includegraphics[width=1.28\linewidth, trim={.5in 2in 1in 1in}, clip]{figs/plotheuristic.pdf}
 
\end{myframe}

\begin{myframe}{Practical Running Time -- Final}

  Average checking time versus puzzle height for 50,000 (8,*)-puzzles.
  \hspace*{-9ex}\includegraphics[width=1.28\linewidth, trim={.5in 2in 1in 1in}, clip]{figs/plotfull.pdf}
 
\end{myframe}


\section{Search}

\againframe{outline}

\begin{myframe}{Search}

  \begin{problem}[SUSP-Search]
    \textbf{Input:} $k \in \NN$ \\[1ex] \textbf{Output:} The maximum
    $s\in \NN$ such that there exists a $(s,k)$-puzzle that is a strong
    uniquely solvable puzzle.
  \end{problem}

  \begin{itemize}
  \item Considered constructive approaches to solving this
    problem that use SUSP-Check as a subroutine.
  \item $(s,k)$-puzzles have $sk$ entries and there are $3^{sk}$ such puzzles.
  \item Even eliminating symmetries, searching the full space for $k >
    4$ is infeasible.
  \item Density of SUSPs quickly approaches 0.
  \item Our approaches are ad hoc and use domain knowledge for heuristics.
  \item SUSP do not form a matroid, augmentation property fails.
  \end{itemize}

  
\end{myframe}

\begin{myframe}{Tree Search}

  \begin{lemma}
    If $P$ is a SUSP and $P' \sse P$, then $P'$ is a SUSP.
  \end{lemma}

  \begin{itemize}
  \item This lemma allows us to construct SUSP from the bottom up.

    \pause
    \medskip
  \item BFS allowed us to explore the set of all SUSP for $k
    \le 5$.
    \begin{itemize}
    \item Implement a sequential desktop version and a parallel
      version to run on Union's $\approx$900-node HPC cluster.
    \item Parallel version used MPI and Map-Reduce to maintain the
      search frontier and support faster verification via lookup.
    \item Searching $k = 5$ originally required the cluster, but
      improvements to the verification algorithm made it
      unnecessary.
    \item Searching $k = 6$ would have exceeded cluster's $32$TB memory.
    \end{itemize}

    \pause
    \medskip
  \item For $k \ge 6$ we implemented ``gradient descent'' for a variety of
    metrics:
    \begin{itemize}
    \item \# of (single, pairs, triples of) rows $P$ could be extended by.
    \item Density of the graph $G_p$.
    \item \# of columns of $P$ which only have two entries from
      $\set{1,2,3}$.
    \item Size of interval spanned by the rows of $P$ in lexicographic
      order.
    \end{itemize}
  \end{itemize}
  
\end{myframe}

\subsection{}

\begin{myframe}{Combining SUSP}
   
  %% \begin{lemma}[CSKU 05, Lemma 6.1]
  %%   Let $P$ be a strong uniquely solvable $(s,k)$-puzzle, then
  %%   $$Q = \condset{ \pi(r_1) \circ\pi(r_2) \circ \ldots \circ \pi(r_s)}{\pi \in \Sym{P}}$$
  %%   is a strong uniquely solvable puzzle.
  %% \end{lemma}

  We've noticed the following behavior of SUSPs under set concatenation:
  
  \begin{obs}[Experimental]
    Let $P_1$ and $P_2$ be \uline{``distinct''} strong uniquely solvable
    puzzles, then
    $$P_1 \circ P_2 = \condset{r_1 \circ r_2}{r_1 \in P_1, r_2 \in
      P_2}$$ is a strong uniquely solvable puzzle.
  \end{obs}

  \begin{itemize}
  \item Here ``distinct'' means that $P_1$ and $P_2$ each decompose
    into the concatenation of pairwise non-equivalent indecomposible
    SUSPs.
  \item Useful for constructing larger puzzles from smaller ones.
  \item No loss in implied $\omega$.
  \end{itemize}


  
\end{myframe}



\subsection{Results}

\begin{myframe}{Strong USP Found -- Examples}

  \tikzoverlay[text width=3cm] at (0cm,0cm) {
    (1,1):\\[.5ex]
    \begin{tabular}{|c|}
      \hline
      1 \\ \hline
    \end{tabular}
  };
  
  \tikzoverlay[text width=3cm] at (0cm,-2cm) {
    (2,2):\\[.5ex]
    \begin{tabular}{|c|c|}
      \hline
      1&3 \\ \hline
      2&1 \\ \hline
    \end{tabular}
  };
  
  \tikzoverlay[text width=3cm] at (0cm,-4cm) {
    (3,3):\\[.5ex]
    \begin{tabular}{|c|c|c|}
      \hline
      1&1&1 \\ \hline
      3&2&1 \\ \hline
      3&3&2 \\ \hline
    \end{tabular}
  };
  
  
  \tikzoverlay[text width=3cm] at (3cm,1.25cm) {
    (5,4):\\[.5ex]
    \begin{tabular}{|c|c|c|c|}
      \hline
      3&1&3&2 \\ \hline
      1&2&3&2 \\ \hline
      1&1&1&3 \\ \hline 
      3&2&1&3 \\ \hline 
      3&3&2&3 \\ \hline
    \end{tabular}
  };
  
  \tikzoverlay[text width=3cm] at (3cm,-1.5cm) {
  (8,5):\\[.5ex]
  \begin{tabular}{|c|c|c|c|c|}
    \hline
    3&3&3&1&1 \\ \hline
    1&1&2&2&1 \\ \hline
    2&1&3&3&2 \\ \hline
    3&2&2&2&3 \\ \hline
    2&1&2&1&3 \\ \hline
    2&2&3&1&2 \\ \hline
    3&2&3&2&1 \\ \hline
    3&1&2&1&1 \\ \hline
  \end{tabular}
  };

  \tikzoverlay[text width=3cm] at (7cm,2cm) {
  (14,6):\\[.5ex]
  \begin{tabular}{|c|c|c|c|c|c|}
    \hline
    2&3&3&1&1&1 \\ \hline
    2&1&1&2&1&1 \\ \hline
    3&3&1&2&1&1 \\\hline
    3&2&2&2&1&1 \\\hline
    2&3&1&1&2&1 \\\hline
    2&2&3&1&2&1 \\\hline
    3&3&1&3&2&1 \\\hline
    3&2&3&3&2&1 \\\hline
    2&1&1&3&1&2 \\\hline
    2&3&1&3&2&2 \\\hline
    3&1&1&1&1&3 \\\hline
    3&3&2&3&1&3 \\\hline
    3&3&2&1&2&3 \\\hline
    2&2&3&2&2&3 \\\hline
  \end{tabular}
  };
  
\end{myframe}

\begin{myframe}{Strong USP Found -- Trends and Comparison}

  \begin{center}
  \begin{tabular}{|c|r|r|r|r|c|}
    \hline
    & \multicolumn{2}{|c|}{[CKSU05]} & \multicolumn{3}{|c|}{This work} \\
    \hline
    Width & Height & $\omega^*$~ & Height & $\omega^*$ & Search Algo~\\
    \hline
    1 & $\le 1$ & & $1=$ & 3.000 & BFS \\
    2 & $\le 3$ & & $2=$ & 2.670 & BFS\\
    3 & $3 \ldots 6$ & 2.642 & $3=$ & 2.642 & BFS\\
    4 & $\le 12$ & & $5=$ & 2.585 & BFS\\
    5 & $\le 24$ & & $8=$ & 2.562 & BFS \\
    6 & $\alert{10} \ldots 45$ & \alert{2.615} &$\alert{14}\le$ & \alert{2.521} & Gradient\\
    7 & $\le 86$ & & $21\le$ & 2.531 & Gradient\\
    8 & $\le 162$ & & $30\le$ & 2.547 & Gradient\\
    9 & $36 \ldots 307$ & 2.592 &$42\le$ & 2.563 & Concat \\
    10 & $\le 581$ & & $64\le$ & 2.562 & Concat \\
    11 & $\le 1098$ & & $112\le$ & 2.540 & Concat\\
    12 & $136 \ldots 2075$ & 2.573 &$196\le$ & 2.521 & Concat \\
    \hline
  \end{tabular}
  \end{center}
  
  \begin{itemize}
  \item $\omega^*$ is the approximate $\omega$ in the limit of
    composing puzzles of these dimensions via direct product.
  \item \;[CKSU05]'s construction asymptotically implies $\omega <
    2.48$.
  \end{itemize}
  
\end{myframe}

\section{Lessons}

\againframe{outline}

\begin{myframe}{Lessons}

  \begin{itemize}
  \item Practical performance $\neq$ worse-case performance.
  \item Problem transformation is effective in theory and in practice.
  \item It's easy to experimentally invalidate specific hypotheses.
  \item It's hard to find patterns in mountains of data.  
  \item It's hard to turn patterns from data into proofs.
  \item Domain knowledge is useful for pruning.
  \item Communication is expensive in HPC. 
  \end{itemize}
  
\end{myframe}

\begin{myframe}{Future Work / Conjectures}

  \begin{conj}
    There is a construction that takes SUSPs of size $(s_1,k_1)$ and
    $(s_2,k_2)$ and produces a $(s_1+s_2,\max(k_1,k_2)+1)$-puzzle that
    is a SUSP.
  \end{conj}
  \begin{itemize}
  \item Would imply $\omega < 2.445$.
  \item Consistent with the SUSP we found for $k = 1\ldots 7$.
  \end{itemize}

  \structure{Search}
  \begin{itemize}
  \item The current bottleneck.
  \item Try iterated local search.
  \item Try repair from concatenated puzzles.
  \item Try to derive better upper bounds.
  \end{itemize}

  \structure{Check}

  \begin{itemize}
  \item Look for reductions with $o(s^3)$ size -- no more 3D matching.
  \item Verify $P$ is SUSP by multiplying random matrices using $P$.
  \end{itemize}
  
  
\end{myframe}

%% ======================================================
%%
%% Bonus Slides
%%
%% ======================================================

\begin{comment}

\section{Bonus Slides}
 
\subsection{Warm-up: Fast Polynomial Multiplication}

\begin{myframe}{Warm-up: Polynomial Multiplication}

  Probably need to skip most of this and next few slides.
  
\end{myframe}

\subsection{Group-Theoretic Matrix Multiplication}

\begin{myframe}{Group-Theoretic Matrix Multiplication}

\end{myframe}

\begin{myframe}{Triple Product Property}

\end{myframe}

\end{comment}




\end{document}
