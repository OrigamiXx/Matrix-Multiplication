%%% Slides without notes
\documentclass[t,10pt,
%handout,
mathserif,xcolor=dvipsnames]{beamer}
\renewcommand{\note}[1]{}

%%% Slides with notes
%% \documentclass[t,11pt,mathserif,xcolor=dvipsnames]{beamer}
%% \usepackage{pgfpages}
%% \setbeameroption{show notes}
%% \setbeameroption{show notes on second screen=right}

%%% Handout with notes
%% \documentclass[t,11pt,handout,mathserif,xcolor=dvipsnames]{beamer} % for printing
%% \usepackage{pgfpages}
%% \setbeameroption{show notes}
%% \setbeameroption{show notes on second screen=right}

%%% Handout without notes
%% \documentclass[t,11pt,handout,mathserif,xcolor=dvipsnames]{beamer}
%% \renewcommand{\note}[1]{}

% includes
\usepackage{beamerthemeMWA}

\usepackage{verbatim}
\usepackage[T1]{fontenc}
\usepackage{ae,aecompl}
\usepackage{amsmath}
\usepackage{soul}
\usepackage{graphicx}
\usepackage{latexsym}
\usepackage{amssymb}
\usepackage{xspace}
\usepackage{amsbsy}
\usepackage{amsthm}
\usepackage{listings}
\usepackage{enumerate}
\usepackage{pgf}
\usepackage{tikz}
\usepgflibrary{arrows}


\lstset{basicstyle=\ttfamily\scriptsize, % print whole listing small
stringstyle=\ttfamily,
keywordstyle=\color{black}\underbar,
sensitive=true,
%identifierstyle=\color{cerulean},
showstringspaces=false}

\renewcommand{\qedsymbol}{\ensuremath{_\blacksquare}}

\setcounter{tocdepth}{4}

\setbeamertemplate{navigation symbols}{}

\usepackage[absolute,overlay]{textpos} 
%\usepackage[colorgrid,texcoord]{eso-pic} 

%adjust the TPHorizModule and TPHorizModule units to the displayed mm %grid 
\TPGrid{210}{297} 



%\usepackage{a4wide}
\usepackage[utf8]{inputenc}
\usepackage{amssymb}
\usepackage{amsmath}
\usepackage{amsthm}
%\usepackage[amsmath,hyperref,thmmarks]{ntheorem}
\usepackage{xspace}
\usepackage{ulem}
\usepackage{graphicx}
\graphicspath{figs}
\usepackage{mdwlist}
\usepackage{color}
\usepackage{soul}
\usepackage{authblk}
\usepackage{algpseudocode}
\usepackage{xparse}
\usepackage{tikz}
%\usepackage{3dplot}
%\usepackage[active,tightpage]{preview}  %generates a tightly fitting border around the work
%\PreviewEnvironment{tikzpicture}
%\setlength\PreviewBorder{2mm}
%% \usetikzlibrary{external}
%% \tikzexternalize
%% \tikzexternaldisable
%% \tikzset{external/force remake=true}
%% \tikzset{external/system call={%
%%   pdflatex \tikzexternalcheckshellescape -halt-on-error -interaction=batchmode -jobname "\image" "\texsource"; 
%%   ps2pdf13 "\image".pdf "\image-13".pdf && cp "\image-13".pdf "\image".pdf}}



\usepackage{changebar}
\setcounter{changebargrey}{0} % darkness of change bars (0 = black)


% Put a period after a paragraph title.
%\let\originalparagraph\paragraph
%\renewcommand{\paragraph}[2][.]{\originalparagraph{#2#1}}


% Margin notes
\newcommand\marginnote[1]{\marginpar{\raggedright\footnotesize\emph{#1}}}

% Symbol to end definitions.
\newcommand{\defnend}{\hfill\ensuremath{\blacktriangleleft}}

% -----------------------------------------------------------------------------
% Theorem environments
% -----------------------------------------------------------------------------

\newtheorem{thm}{Theorem}
\newtheorem{lem}[thm]{Lemma}
\newtheorem{question}[thm]{Question}
\newtheorem{conj}[thm]{Conjecture}
\newtheorem{claim}[thm]{Claim}
\newtheorem{cor}[thm]{Corollary}
\newtheorem{obs}[thm]{Observation}
\newtheorem{prop}[thm]{Proposition}


%% \newtheorem{thm}{Theorem}
%% \newtheorem{lem}[thm]{Lemma}
%% \newtheorem{question}[thm]{Question}
%% \newtheorem{conj}[thm]{Conjecture}
%% \newtheorem{fact}[thm]{Fact}
%% \newtheorem{claim}[thm]{Claim}
%% \newtheorem{cor}[thm]{Corollary}
%% \newtheorem{definition}[thm]{Definition}
%% \newtheorem{remark}[thm]{Remark}
%% \newtheorem{prop}[thm]{Proposition}
%% \newtheorem{obs}[thm]{Observation}

\newcommand{\mathcolor}{\color{ta2gray}}
\newcommand{\inl}[1]{$\mathcolor #1$}
\newcommand{\stress}[1]{\emph{\structure{#1}}}
\newcommand{\cit}[1]{\textbf{\structure{\small  #1}}}
\newcommand{\IFPC}{\logic{FPC}}
\newcommand{\titl}[1]{\textbf{\structure{\small  #1}}}
\newcommand{\displaym}[1]{$$\mathcolor #1 $$}
\newcommand{\DLT}{\ensuremath{\mathrm{DLogTime}}\xspace}
\newcommand{\Part}{\mathcal{P}}
\newcommand{\E}{\ensuremath{\mathcal{E}}}
\newcommand{\orb}{\mathrm{Orb}}
\newcommand{\Aa}{\ensuremath{\mathbb{A}}}
\newcommand{\ra}{\rightarrow}


\newcommand{\F}{\ensuremath{\mathbb{F}}}
\newcommand{\de}{:=}
\newcommand{\set}[1]{\ensuremath{\{{#1}\}}}
\newcommand{\condset}[2]{\ensuremath{\set{{#1}\;|\;{#2}}}}
\newcommand{\xs}{\ensuremath{x_1,\ldots,x_n}}
\newcommand{\G}{\ensuremath{\mathcal{G}}\xspace}
\newcommand{\X}{\ensuremath{\mathcal{X}}\xspace}
\newcommand{\p}{\partial}
\newcommand{\V}[1]{\ensuremath{\mathrm{var}({#1})}}
\newcommand{\rk}{\mathrm{rank}}
\newcommand{\D}{\mathrm{Det}}
\newcommand{\nev}{\not\equiv}
\newcommand{\NN}{\ensuremath{\mathbb{N}}\xspace}
\newcommand{\bs}{\ensuremath{\backslash}}
%\newcommand{\C}{\ensuremath{\mathcal{C}}\xspace}
\newcommand{\M}{\ensuremath{\mathcal{M}}\xspace}
\newcommand{\U}{\ensuremath{\mathcal{U}}\xspace}
\newcommand{\Op}{\ensuremath{\mathcal{O}}\xspace}
\newcommand{\s}{\ensuremath{\mathcal{S}}\xspace}
\newcommand{\ev}{\equiv}
\newcommand{\la}{\leftarrow}
\newcommand{\B}{\mathbb{B}}
\newcommand{\C}{\mathbb{C}}
\newcommand{\A}{\mathcal{A}}



\newcommand{\Sym}[1]{\ensuremath{\mathrm{Sym}_{#1}}\xspace}
\newcommand{\Stab}{\ensuremath{\mathrm{Stab}}\xspace}

\newcommand{\poly}{\ensuremath{\mathrm{poly}}\xspace}

\newcommand{\fin}{\mathrm{fin}}
\newcommand{\logic}[1]{\text{\upshape #1}\xspace}
\newcommand{\dom}{\ensuremath{\mathrm{dom}}}
\newcommand{\AC}[1]{\ensuremath{\mathrm{AC}^{#1}}\xspace}
\newcommand{\NC}[1]{\ensuremath{\mathrm{NC}^{#1}}\xspace}
\newcommand{\FO}{\logic{FO}}
\newcommand{\FOARB}{\ensuremath{(\FO+\mathrm{arb})}\xspace}
\newcommand{\FOA}{\ensuremath{(\FO\mathrm{+arith)}}\xspace}
\newcommand{\FOa}[1]{\ensuremath{\FO(#1)}\xspace}
\newcommand{\CPTC}{\logic{$\tilde{\text C}$PT(Card)}}
\newcommand{\NUM}{\ensuremath{\mathrm{NUMBER}}}
\newcommand{\FOC}{\ensuremath{\FO\mathrm{+C}}\xspace}
\newcommand{\PT}{\ensuremath{\logic{P}}\xspace}
\newcommand{\Pp}{\ensuremath{\PT/\poly}\xspace}
\newcommand{\BPP}{\ensuremath{\mathrm{BPP}}\xspace}
\newcommand{\NP}{\ensuremath{\logic{NP}}\xspace}
\newcommand{\LFP}{\logic{LFP}}
\newcommand{\LFPC}{\logic{LFPC}}
\newcommand{\FP}{\logic{FP}}
\newcommand{\FPR}{\logic{FPR}}
\newcommand{\FF}[1]{\mathbb{F}_{#1}}

\newcommand{\FPC}{\logic{FPC}}
\newcommand{\Orb}[2]{\ensuremath{\mathrm{Orb}_{#1}({#2})}\xspace}
\newcommand{\OP}[2]{\ensuremath{{#1}\text{-OP}_{#2}\xspace}}
\renewcommand{\implies}{\Rightarrow}
\newcommand{\qd}{\quad}
\newcommand{\SP}{\mathrm{Sup}}
\newcommand{\spt}{\mathrm{spt}}
\newcommand{\sse}{\subseteq}
\newcommand{\ssn}{\subsetneq}
\newcommand{\spe}{\supseteq}
\newcommand{\spn}{\supsetneq}
\newcommand{\es}{\emptyset}
\newcommand{\inv}{^{-1}}
%\newcommand{\so}{{\ensuremath{\langle S\rangle}\xspace}}
%\newcommand{\ls}{\le_\so}
\newcommand{\Perm}[1]{\ensuremath{\mathrm{Per}({#1})}\xspace}

\newcommand{\com}[1]{\noindent\fbox{\hl{#1}}}
\renewcommand{\bar}[1]{\overline{#1}}

\newcommand{\LT}[1]{\ensuremath{\mathrm{LT}({#1})}\xspace}
\newcommand{\LM}[1]{\ensuremath{\mathrm{LM}({#1})}\xspace}

\newcommand{\FOt}{\ensuremath{\FO(\tau)}\xspace}
\newcommand{\LFPCt}{\ensuremath{(\LFPC)(\tau)}\xspace}

\newcommand{\hdef}[2]{\structure{#1}\; #2}

\newcommand{\alignedeq}[1]{
\begin{equation*}
\begin{aligned}
#1
\end{aligned}
\end{equation*}
}

\newcommand{\deffunc}[3]{
\begin{flalign*}
{#1} :&~#2 \\
    &~#3
\end{flalign*}
}

\newcommand{\deffunca}[3]{
{#1} :&~#2 \\
    &~#3 \\
}

\newcommand{\alg}[2]{
\hrule\smallskip
\noindent{#1}
\smallskip
\hrule
\vspace{-1.5ex}
\begin{enumerate*}
{#2}
\end{enumerate*}
\vspace{-1.5ex}
\hrule
}

\newcommand{\algio}[4]{
  \alg{
    {#1}
    \smallskip
    \hrule
    \smallskip
    \hspace*{2ex}Input: {#2} \\
    \hspace*{2ex}Output: {#3} 
  }{
    {#4}
  }
}


%%%%%%% Frame template %%%%%%%%%%

%% \begin{myframe}{ }
%%   \begin{itemize}
%%   \item
%%   \end{itemize}
%% \end{myframe}

%%%%%%% End Frame template %%%%%

%\setbeamertemplate{footline}[page number]

\newcommand{\QQ}{\ensuremath{\mathbb{Q}}}

\title{\textbf{Computer-Aided Search for \\ Matrix Multiplication Algorithms}}

\institute{{\large \structure{Matthew Anderson} \quad  Zongliang Ji \quad  Anthony Yang Xu \\ \includegraphics[width=1.3in]{figs/uc_logo.eps}}}
\date{December 13, 2017 \\[1ex] {\small Simons Institute for the Theory of Computing}}

\begin{document}
\begin{frame}[plain]
  
  \vspace{5ex}

  \titlepage

\end{frame}

\section{Introduction}

\subsection{The Matrix Multiplication Problem}

\begin{myframe}{Matrix Multiplication}

  \begin{problem}%[Square Matrix Multiplication]
    \textbf{Input:} $A \in \F^{n \times n}$, $B \in \F^{n \times n}$ \\
    \textbf{Output:} $C = A \times B \in \F^{n\times n}$.    
  \end{problem}

  For example:
  $$
  \left[
    \begin{tabular}{cc}
      1 & 2 \\
        2 & 0
    \end{tabular}
    \right]
  \times
  \left[
    \begin{tabular}{cc}
      -1 & 3 \\
      1 & 1
    \end{tabular}
    \right]
  =
  \left[
    \begin{tabular}{cc}
      1 & 5 \\
      -2 & 6
    \end{tabular}
    \right]    
  $$
  
  
  %\item Focus on \structure{square} matrix multiplication where $m = n = p$.
  How many operations does it take to multiply two $n$-by-$n$
  matrices?
  \begin{itemize}
  \item $O(n^3)$ by naively computing $n^2$ dot products of rows of $A$ and columns of $B$.
  \item $\Omega(n^2)$ because there are at $n^2$ cells to output.
  \end{itemize}
  \begin{question}
  What is the smallest $\omega \le 3$ such that $n$-by-$n$ matrix
  multiplication can be done in time $O(n^\omega)$?
  \end{question}
    
\end{myframe}

\subsection{Historical Context}

\begin{myframe}{Progress on $\omega$}

  \begin{center}
  {\Large
  \begin{tabular}{ll}
    \structure{$3$} & Naive \\
    \structure{$\uline{2}.808$} & Strassen 1969 \\
    \structure{$2.\uline{7}96$} & Pan 1978 \\
    \structure{$2.7\uline{8}$} & Bini et al 1979 \\
    \structure{$2.\uline{5}22$} & Schönhage 1981 \\
    \structure{$2.\uline{4}96$} & Coppersmith \& Winograd 1982 \\
    \structure{$2.4\uline{7}9$} & Strassen 1986 \\  % Laser
    \structure{$2.\uline{3}75477$} & Coppersmith \& Winograd 1987 \\
    \structure{$2.37\uline{4}$} & Stothers 2010 \\
    \structure{$2.37\uline{2}8642$} & Williams 2011 \\
    \structure{$2.37286\uline{3}9$} & François Le Gall 2014
  \end{tabular}
  }
  \end{center}
  
\end{myframe}

\begin{frame}[label=outline]{\vspace{1mm}\textbf{Outline}}

  \begin{itemize}
  \item Introduction
  \item Cohn-Umans Framework
  \item Verification
  \item Search
  \item Lessons
  \end{itemize}
  
\end{frame}

\section{Cohn-Umans Framework}

\begin{myframe}{Cohn-Umans Framework}

\end{myframe}

\subsection{Warm-up: Fast Polynomial Multiplication}

\begin{myframe}{Warm-up: Polynomial Multiplication}

  Probably need to skip most of this and next few slides.
  
\end{myframe}

\subsection{Group-Theoretic Matrix Multiplication}

\begin{myframe}{Group-Theoretic Matrix Multiplication}

\end{myframe}

\begin{myframe}{Triple Product Property}

\end{myframe}

\subsection{Strong Uniquely Solvable Puzzles}

\begin{myframe}{Puzzles}

  \begin{definition}[Puzzle]
    An $(s,k)$-\emph{\structure{puzzle}} is a subset $P \sse U_k = \set{1,2,3}^k$ with $|P| = s$.
  \end{definition}

  Consider $P = \set{(1,2,3), (2, 3, 2)} \sse \set{1,2,3}^3$.
  
  \begin{itemize}
  \item $P$ is a (2,3)-puzzle.
  \item $P$ has two \emph{rows} $(1,2,3)$ and $(2,3,2)$.
  \item $P$ has three \emph{columns}.
  \end{itemize}

  Note that:
  \begin{itemize}
  \item The columns are ordered.
  \item The rows are unordered (as $P$ is a set).
  \item It sometimes helps to consider elements of $U_k$ as ordered
    lexicographically.
  \end{itemize}

  XXX - insert diagram to right and format.
  
\end{myframe}

\begin{myframe}{\uline{Uniquely Solvable} Puzzles -- Intuition}
  
  It helps to think of each row of a puzzle as consisting of
  \structure{three} pieces corresponding to the coordinates of the row
  that are 1, 2, or 3.


  \begin{itemize}
  \item E.g., the row $(1, 2, 1, 1)$ consists of a 1-piece $(1, *, 1,
    1)$, a 2-piece $(*, 2, *, *)$ and a 3-piece $(*, *, *, *)$.
    
  \item A 1-piece, 2-piece, and 3-piece can be fit together to form a
    row iff there is exactly one non-$*$ for each coordinate among the
    three pieces.
  
  \item E.g., pieces $(1, 1, *, *)$, $(*, *, 2, *)$, $(*, 3, 3, *)$
    cannot be fix together because they overlap on the $2^{nd}$ and
    $3^{rd}$ coords, and there is no entry on the $4^{th}$.

  \end{itemize}

  Informally, a puzzle $P$ is \emph{\structure{uniquely solvable}} if
  there is no way to reorganize the 1-, 2-, 3-pieces of $P$ into a
  puzzle different from $P$.

  \begin{itemize}
  \item This is a natural property that holds of ``good'' real-world puzzles:
    \begin{itemize}
    \item jigsaw puzzles (locally), and
    \item sudoku puzzles (globally).
    \end{itemize}
  \end{itemize}
  
\end{myframe}


\begin{myframe}{Uniquely Solvable Puzzle -- Intuition}

  XXX - diagram of uniquely solvable, and not.
  
\end{myframe}


\begin{myframe}{Uniquely Solvable Puzzle -- Formal}
  
  \begin{definition}[Uniquely Solvable Puzzle]
    An $(s,k)$-puzzle $P$ is called \emph{\structure{uniquely solvable}} if
    $\forall \pi_1, \pi_2, \pi_3 \in \Sym{P}:$
    \begin{enumerate}
    \item either $\pi_1 = \pi_2 = \pi_3$, or
    \item $\exists r \in P, \exists i \in [k]$ such that \structure{at least} two
      of the following hold:
      \begin{enumerate}
      \item $(\pi_1(r))_i = 1$,
      \item $(\pi_2(r))_i = 2$,
      \item $(\pi_3(r))_i = 3$.
      \end{enumerate}
    \end{enumerate}
    
  \end{definition}

  Basically, for every way of non-trivial way of reordering the 1-,
  2-, 3-pieces according to $\pi_1, \pi_2, \pi_3$, they cannot all fit
  together.
    
  
\end{myframe}

\begin{myframe}{\uline{Strong} Uniquely Solvable Puzzles}

  \begin{definition}[Strong Uniquely Solvable Puzzle]

    An $(s,k)$-puzzle $P$ is called \emph{\structure{strong uniquely solvable}} if
    $\forall \pi_1, \pi_2, \pi_3 \in \Sym{P}:$
    \begin{enumerate}
    \item either $\pi_1 = \pi_2 = \pi_3$, or
    \item $\exists r \in P, \exists i \in [k]$ such that \structure{exactly} two
      of the following hold:
      \begin{enumerate}
      \item $(\pi_1(r))_i = 1$,
      \item $(\pi_2(r))_i = 2$,
      \item $(\pi_3(r))_i = 3$.
      \end{enumerate}
    \end{enumerate}

  \end{definition}

  No good physical intuition for the ``exactly two'' part.
    
  
\end{myframe}

\begin{myframe}{Properties of Strong USPs -- I}

\end{myframe}

\begin{myframe}{Properties of Strong USPs -- II}

\end{myframe}

\begin{myframe}{Our Goals \& Approach}

\end{myframe}

\section{Verification}

\againframe{outline}

\subsection{Brute Force}

\begin{myframe}{Brute Force}

\end{myframe}

\subsection{Pruning}

\begin{myframe}{Pruning}

\end{myframe}

\subsection{Reduction to 3D Matching}

\begin{myframe}{Reduction to 3D Matching -- I}

Remark on symmetry
  
\end{myframe}

\begin{myframe}{Reduction to 3D Matching -- II}

\end{myframe}

\subsubsection{Bidirectional Search}

\begin{myframe}{Bidirectional Search}

\end{myframe}

\subsubsection{Reduction to SAT}

\begin{myframe}{Reduction to SAT}

\end{myframe}

\subsubsection{Reduction to IP}

\begin{myframe}{Reduction to IP}

\end{myframe}

\subsection{Results}

\begin{myframe}{Results -- I}

  Generate data?
  Heuristics?
  
\end{myframe}

\begin{myframe}{Results -- II}

\end{myframe}


\section{Search}

\againframe{outline}

\subsection{BFS \& DFS}

\begin{myframe}{BFS \& DFS}

\end{myframe}

\subsection{Bidirectional Search}

\begin{myframe}{Bidirection Search}

\end{myframe}

\subsection{Parallel BFS + DP Querying}

\begin{myframe}{Parallel BFS}

\end{myframe}

\begin{myframe}{Parallel BFS + DP Querying}

\end{myframe}

\subsection{Greedy}

\begin{myframe}{Greedy}

\end{myframe}

\subsection{Results}

\begin{myframe}{Results -- I}

  Examples + table
  
\end{myframe}

\begin{myframe}{Results -- II}

\end{myframe}

\section{Lessons}

\againframe{outline}

\begin{myframe}{Lessons}
 
\end{myframe}

\begin{myframe}{Future Work}

\end{myframe}

%% ======================================================
%%
%% Bonus Slides
%%
%% ======================================================

\section{Bonus Slides}
 






\end{document}
