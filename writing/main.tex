\documentclass[11pt]{article}
\usepackage{fullpage}
\usepackage{rpmacros}
\RequirePackage[colorlinks=true]{hyperref}
\hypersetup{
  linkcolor=[rgb]{0,0,0.4},
  citecolor=[rgb]{0, 0.4, 0},
  urlcolor=[rgb]{0.6, 0, 0}
}
\usepackage{mathpazo}
\usepackage{bbm}
\usepackage{todonotes}
\usepackage{lipsum}
\usepackage{setspace}
\usepackage{mdframed}
\usepackage{tikz}
\usetikzlibrary{decorations.pathreplacing,backgrounds}

\usepackage[font=footnotesize]{caption}

\usepackage{amsthm}
\usepackage{thmtools,thm-restate}

\numberwithin{equation}{section}
\declaretheoremstyle[bodyfont=\it,qed=\qedsymbol]{noproofstyle} 

\declaretheorem[numberlike=equation]{axiom}

\declaretheorem[numberlike=equation]{observation}
\declaretheorem[numberlike=equation,style=noproofstyle,name=Observation]{observationwp}
\declaretheorem[name=Observation,numbered=no]{observation*}

\declaretheorem[numberlike=equation]{fact}
\declaretheorem[numberlike=equation]{subclaim}
\declaretheorem[numberlike=equation]{problem}

\declaretheorem[numberlike=equation]{theorem}
\declaretheorem[numberlike=equation,style=noproofstyle,name=Theorem]{theoremwp}
\declaretheorem[name=Theorem,numbered=no]{theorem*}

\declaretheorem[numberlike=equation]{lemma}
\declaretheorem[name=Lemma,numbered=no]{lemma*}
\declaretheorem[numberlike=equation,style=noproofstyle,name=Lemma]{lemmawp}

\declaretheorem[numberlike=equation]{corollary}
\declaretheorem[name=Corollary,numbered=no]{corollary*}
\declaretheorem[numberlike=equation,style=noproofstyle,name=Corollary]{corollarywp}

\declaretheorem[numberlike=equation]{proposition}
\declaretheorem[name=Proposition,numbered=no]{proposition*}
\declaretheorem[numberlike=equation,style=noproofstyle,name=Proposition]{propositionwp}

\declaretheorem[numberlike=equation]{claim}
\declaretheorem[name=Claim,numbered=no]{claim*}
\declaretheorem[numberlike=equation,style=noproofstyle,name=Claim]{claimwp}

\declaretheorem[numberlike=equation]{conjecture}
\declaretheorem[name=Conjecture,numbered=no]{conjecture*}

\declaretheorem[numberlike=equation]{question}
\declaretheorem[name=Question,numbered=no]{question*}

\declaretheorem[name=Open Problem]{openproblem}
\declaretheorem[name=Exercise]{exercise}

\declaretheoremstyle[bodyfont=\it,qed=$\lozenge$]{defstyle} 

\declaretheorem[numberlike=equation,style=defstyle]{definition}
\declaretheorem[unnumbered,name=Definition,style=defstyle]{definition*}

\declaretheorem[numberlike=equation,style=defstyle]{example}
\declaretheorem[unnumbered,name=Example,style=defstyle]{example*}

\declaretheorem[numberlike=equation,style=defstyle]{notation}
\declaretheorem[unnumbered,name=Notation=defstyle]{notation*}

\declaretheorem[numberlike=equation,style=defstyle]{construction}
\declaretheorem[unnumbered,name=Construction,style=defstyle]{construction*}

\declaretheorem[numberlike=equation,style=defstyle]{remark}
\declaretheorem[unnumbered,name=Remark,style=defstyle]{remark*}

\renewcommand{\subsectionautorefname}{Section}
\renewcommand{\sectionautorefname}{Section}
\newcommand{\algorithmautorefname}{Algorithm}



%%% Local Variables: 
%%% mode: latex
%%% TeX-master: "main"
%%% End: 

\usepackage{nth}
\usepackage{intcalc}
\usepackage{etoolbox}
\usepackage{xstring}
\hypersetup{
%breaklinks=true % this is to handle linbreaks in url if latex+dvips+ps2pdf used
                % Thanks to Ben Lee Volk for this fix
}

\usepackage{ifpdf}
\ifpdf
\else
\usepackage[quadpoints=false]{hypdvips}
\fi

\newcommand{\mfbiberr}[1]{\PackageWarning{miforbes-bibtex}{#1}}

\newcommand{\cSTOC}[1]{\nth{\intcalcSub{#1}{1968}}\ Annual\ ACM\ Symposium\ on\ Theory\ of\ Computing\ (STOC\ #1)}
\newcommand{\cFSTTCS}[1]{\nth{\intcalcSub{#1}{1980}}\ International\ Conference\ on\ Foundations\ of\ Software\ Technology\ and\ Theoretical\ Computer\ Science\ (FSTTCS\ #1)}
\newcommand{\cCCC}[1]{\ifnumcomp{#1}{<}{1996}{%1985-1995
\nth{\intcalcSub{#1}{1985}}\ Annual\ Structure\ in\ Complexity\ Theory\ Conference\ (Structures\ #1)}%
{\ifnumcomp{#1}{<}{2015}{%1996-2014
\nth{\intcalcSub{#1}{1985}}\ Annual\ IEEE\ Conference\ on\ Computational\ Complexity (CCC\ #1)}%
{%2015-
\nth{\intcalcSub{#1}{1985}}\ Annual\ Computational\ Complexity\ Conference\ (CCC\ #1)}}}
\newcommand{\cFOCS}[1]{\ifnumcomp{#1}{<}{1966}{%1960-1965
\nth{\intcalcSub{#1}{1959}}\ Annual\ Symposium\ on\ Switching\ Circuit\ Theory\ and\ Logical\ Design (SWCT #1)}%
{\ifnumcomp{#1}{<}{1975}{%1966-1974
\nth{\intcalcSub{#1}{1959}}\ Annual\ Symposium\ on\ Switching\ and\ Automata\ Theory\ (SWAT #1)}%
{%1975--
\nth{\intcalcSub{#1}{1959}}\ Annual\ IEEE\ Symposium\ on\ Foundations\ of\ Computer\ Science\ (FOCS\ #1)}}}
\newcommand{\cRANDOM}[1]{\nth{\intcalcSub{#1}{1996}}\ International\ Workshop\ on\ Randomization\ and\ Computation\ (RANDOM\ #1)}
\newcommand{\cISSAC}[1]{#1\ International\ Symposium\ on\ Symbolic\ and\ Algebraic\ Computation\ (ISSAC\ #1)}
\newcommand{\cICALP}[1]{\nth{\intcalcSub{#1}{1973}}\ International\ Colloquium\ on\ Automata,\ Languages and\ Programming\ (ICALP\ #1)}
\newcommand{\cCOLT}[1]{\nth{\intcalcSub{#1}{1987}}\ Annual\ Conference\ on\ Computational\ Learning\ Theory\ (COLT\ #1)}
\newcommand{\cCSR}[1]{\nth{\intcalcSub{#1}{2005}}\ International\ Computer\ Science\ Symposium\ in\ Russia\ (CSR\ #1)}
\newcommand{\cMFCS}[1]{\nth{\intcalcSub{#1}{1975}}\ Internationl\ Symposium\ on\ the\ Mathematical\ Foundations\ of\ Computer\ Science\ (MFCS\ #1)}
\newcommand{\cEUROSAM}[1]{International\ Symposium\ on\ Symbolic\ and\ Algebraic\ Computation\ (EUROSAM\ #1)}
\newcommand{\cSODA}[1]{\nth{\intcalcSub{#1}{1989}}\ Annual\ ACM-SIAM\ Symposium\ on\ Discrete\ Algorithms\ (SODA\ #1)}
\newcommand{\cSTACS}[1]{\nth{\intcalcSub{#1}{1983}}\ Symposium\ on\ Theoretical\ Aspects\ of\ Computer\ Science\ (STACS\ #1)}



\newcommand{\pSTOC}[1]{Preliminary\ version\ in\ the\ \emph{\cSTOC{#1}}}
\newcommand{\pFSTTCS}[1]{Preliminary\ version\ in\ the\ \emph{\cFSTTCS{#1}}}
\newcommand{\pCCC}[1]{Preliminary\ version\ in\ the\ \emph{\cCCC{#1}}}
\newcommand{\pFOCS}[1]{Preliminary\ version\ in\ the\ \emph{\cFOCS{#1}}}
\newcommand{\pRANDOM}[1]{Preliminary\ version\ in\ the\ \emph{\cRANDOM{#1}}}
\newcommand{\pISSAC}[1]{Preliminary\ version\ in\ the\ \emph{\cISSAC{#1}}}
\newcommand{\pICALP}[1]{Preliminary\ version\ in\ the\ \emph{\cICALP{#1}}}
\newcommand{\pCOLT}[1]{Preliminary\ version\ in\ the\ \emph{\cCOLT{#1}}}
\newcommand{\pCSR}[1]{Preliminary\ version\ in\ the\ \emph{\cCSR{#1}}}
\newcommand{\pMFCS}[1]{Preliminary\ version\ in\ the\ \emph{\cMFCS{#1}}}
\newcommand{\pEUROSAM}[1]{Preliminary\ version\ in\ the\ \emph{\cEUROSAM{#1}}}
\newcommand{\pSODA}[1]{Preliminary\ version\ in\ the\ \emph{\cSODA{#1}}}
\newcommand{\pSTACS}[1]{Preliminary\ version\ in\ the\ \emph{\cSTACS{#1}}}


\newcommand{\STOC}[1]{Proceedings\ of\ the\ \cSTOC{#1}}
\newcommand{\FSTTCS}[1]{Proceedings\ of\ the\ \cFSTTCS{#1}}
\newcommand{\CCC}[1]{Proceedings\ of\ the\ \cCCC{#1}}
\newcommand{\FOCS}[1]{Proceedings\ of\ the\ \cFOCS{#1}}
\newcommand{\RANDOM}[1]{Proceedings\ of\ the\ \cRANDOM{#1}}
\newcommand{\ISSAC}[1]{Proceedings\ of\ the\ \cISSAC{#1}}
\newcommand{\ICALP}[1]{Proceedings\ of\ the\ \cICALP{#1}}
\newcommand{\COLT}[1]{Proceedings\ of\ the\ \cCOLT{#1}}
\newcommand{\CSR}[1]{Proceedings\ of\ the\ \cCSR{#1}}
\newcommand{\MFCS}[1]{Proceedings\ of\ the\ \cMFCS{#1}}
\newcommand{\EUROSAM}[1]{Proceedings\ of\ the\ \cEUROSAM{#1}}
\newcommand{\SODA}[1]{Proceedings\ of\ the\ \cSODA{#1}}
\newcommand{\STACS}[1]{Proceedings\ of\ the\ \cSTACS{#1}}

\newcommand{\arXiv}[1]{\texttt{\href{http://arxiv.org/abs/#1}{arXiv:#1}}}
\newcommand{\farXiv}[1]{Full\ version\ at\ \arXiv{#1}}
\newcommand{\parXiv}[1]{Preliminary\ version\ at\ \arXiv{#1}}

\newcommand{\ECCCurl}[2]{http://eccc.hpi-web.de/report/\ifnumcomp{#1}{>}{93}{19}{20}#1/#2/}
\newcommand{\cECCC}[2]{\href{\ECCCurl{#1}{#2}}{Electronic\ Colloquium\ on\ Computational\ Complexity\ (ECCC),\ Technical\ Report\ TR#1-#2}}
\newcommand{\shortECCC}[2]{\texttt{\href{http://eccc.hpi-web.de/report/\ifnumcomp{#1}{>}{93}{19}{20}#1/#2/}{eccc:TR#1-#2}}}
\newcommand{\ECCC}{Electronic\ Colloquium\ on\ Computational\ Complexity\ (ECCC)}
\newcommand{\fECCC}[2]{Full\ version\ in\ the\ \cECCC{#1}{#2}}
\newcommand{\pECCC}[2]{Preliminary\ version\ in\ the\ \cECCC{#1}{#2}}

\newcommand{\parseECCC}[1]{% Takes a string of the form TRxx/xxx or
%                          % TRxx-xxx and returns short ECCC link
\StrSubstitute{#1}{TR}{}[\tmpstring]%
\IfSubStr{\tmpstring}{/}{ %assuming string is of the form TRxx/xxx
\StrBefore{\tmpstring}{/}[\ecccyear]%
\StrBehind{\tmpstring}{/}[\ecccreport]%
}{% assuming string is of the form TRxx-xxx
\StrBefore{\tmpstring}{-}[\ecccyear]%
\StrBehind{\tmpstring}{-}[\ecccreport]%
}%
\shortECCC{\ecccyear}{\ecccreport}}


\usepackage{algorithmicx}
\usepackage{algorithm} % http://ctan.org/pkg/algorithms
\usepackage{algpseudocode}
\renewcommand{\algorithmicensure}{\textbf{Output:}}
\renewcommand{\algorithmicrequire}{\textbf{Input:}}

\algrenewcommand\algorithmicindent{1.0em}%

\newcommand{\RPnote}[1]{\textcolor{BrickRed}{RP: #1}}
\newcommand{\Mattnote}[1]{\textcolor{OliveGreen}{MWA: #1}}
\newcommand{\BLnote}[1]{\textcolor{Blue}{BLV: #1}}
\newcommand{\Anote}[1]{\textcolor{Plum}{A: #1}}
\newcommand{\MFnote}[1]{\textcolor{DarkOrchid}{MF: #1}}

\newcommand*\samethanks[1][\value{footnote}]{\footnotemark[#1]}
\newcommand\sse{\subseteq}
\newcommand\Sym[1]{\ensuremath{\mathrm{Sym}_{#1}}}

%\onehalfspacing
\date{}

\title{Matrix Multiplication: Finding Strong Uniquely-Solvable Puzzles
{\IfFileExists{./sha.tex}{\\\small SHA: \input{sha}}{}}}
\author{
Matthew Anderson\thanks{Department of Computer Science, Union College, Schenectady, New York, USA, E-mails: \texttt{andersm2@union.edu, jiz@union.edu, xua@union.edu}}%
\and%
Zongliang Ji\samethanks[1]
\and%
Anthony Yang Xu\samethanks[1]
}
\begin{document}
\maketitle

\begin{abstract}

\end{abstract}

\thispagestyle{empty}
\newpage
\pagenumbering{arabic}


\section{Introduction}
\label{sec:intro}

Cohn and Umans \cite{cu03} introduced an approach for developing
faster algorithms for matrix multiplication by reducing this question
to a search for group-theoretic groups that satisfy a certain
combinatorial property.  

\section{Preliminaries}
\label{sec:prelim}

\newcommand\ordset[1]{\ensuremath{[#1]}}

We use $\ordset{n}$ to denote the set $\set{0,1,2,\ldots, n-1}$.
\cite{cksu05} introduced the idea of a \emph{puzzle}.  For a set $Q$,
$\Sym{Q}$ denotes the symmetric group on the elements of $Q$.

\begin{definition}[Puzzle]
  For $s, k \in \Natural$, an $(s,k)$-\emph{puzzle} is a
  subset $P \sse U_k = \ordset{3}^k$ with $|P| = s$.
\end{definition}

We say that an $(s,k)$-puzzle has $s$ rows and $k$ columns.  The
columns are inherent ordered and indexed by $\ordset{k}$.  The rows are not
inherently ordered, however, it is often convienent to assume that the
rows are arbitrarily ordered and indexed by $\ordset{s}$.

\cite{cksu05} also establish a particular combinatorial property of
such puzzles that can derive groups that matrix multiplication can be
embedded into.  Such puzzles are called \emph{strong} uniquely
solvable puzzles.  However, to give some intuition we first explain a
simpler version of the property called \emph{uniquely solvable
  puzzles}.

\begin{definition}[Uniquely Solvable Puzzle (USP)]
  ~\\An $(s,k)$-puzzle $P$ is \emph{uniquely solvable} if
  $\forall \pi_0, \pi_1, \pi_2 \in \Sym{P}:$
  \begin{enumerate}
  \item either $\pi_0 = \pi_1 = \pi_2$, or
  \item $\exists r \in P, \exists i \in \ordset{k}$ such that at least two
    of the following hold:
    \begin{enumerate}
    \item $(\pi_0(r))_i = 0$,
    \item $(\pi_1(r))_i = 1$,
    \item $(\pi_2(r))_i = 2$.
    \end{enumerate}
  \end{enumerate}
\end{definition}

Informally a puzzle is \textbf{not} uniquely solvable if each row of
the puzzle can be broken into zeros, ones, and twos pieces and then
the rows can be reassembled in a different way so that each new row is
a combination of a zeroes, a ones, and twos piece where there is
exactly element of $\ordset{3}$ for each column.  Observe that uniquely
solvable puzzles can have at most $2^k$ rows because each zeroes
piece, ones piece, and two piece must be unique, as otherwise the
duplicate pieces can be swapped making the puzzle not uniquely
solvable.  The definition of \emph{strong} uniquely solvable puzzle is
below, it is nearly the same except that it requires that there be a
collision on a column between exactly two pieces, not two or more
pieces like in the original definition.

\begin{definition}[Strong Uniquely Solvable Puzzle]
  ~\\
  An $(s,k)$-puzzle $P$ is \emph{strong uniquely solvable} if
  $\forall \pi_0, \pi_1, \pi_2 \in \Sym{P}:$
  \begin{enumerate}
  \item either $\pi_0 = \pi_1 = \pi_2$, or
  \item $\exists r \in P, \exists i \in \ordset{k}$ such that exactly two
    of the following hold:
    \begin{enumerate}
    \item $(\pi_0(r))_i = 0$,
    \item $(\pi_1(r))_i = 1$,
    \item $(\pi_2(r))_i = 2$.
    \end{enumerate}
  \end{enumerate}
  
\end{definition}

Observe that the properties of uniquely solvable and strong uniquely
solvable are invariant to the ordering of the rows or columns of a
puzzle.  This fact is used implicitly.  Also note that these
properties are invariant to maps between puzzles induced by
permutations from \Sym{\ordset{3}} on the puzzle cell elements.

Cohn et al.~ show the following connection between the existence of
strong uniquely solvable puzzles and upper bounds on the exponent of
matrix multiplication $\omega$.

\begin{lemma}[{\cite[Corollary 3.6]{cksu05}}]
  If there is a strong uniquely solvable $(s,k)$-puzzle,
  $$\omega \le \min_{m \ge 3, m \in \Natural} \frac{3 \log
    m}{\log(m-1)} - \frac{3 \log s!}{sk \log(m-1)}.$$
\end{lemma}

In the same article, the authors also demonstrate the existence of an
infinite family of strong uniquely solvable puzzles that achieves a
non-trivial bound on $\omega$.

\begin{lemma}[{\cite[Proposition 3.8]{cksu05}}]
  There is an infinite family of strong uniquely solvable puzzles that
  achieves $\omega < 2.48$.
\end{lemma}

Finally, they conjecture that strong uniquely solvable puzzles provide
a root to achieving quadratic time matrix multiplication.

\begin{conjecture}[{\cite{cksu05}}]
  There exists a family of strong uniquely solvable puzzles that
  implies $\omega = 2$.
\end{conjecture}

Unfortunately, this conjecture was recently shown to be false.

\begin{lemma}[\cite{bccgu16}]
  Strong uniquely solvable puzzles cannot show $\omega < 2 +
  \epsilon$, for some $\epsilon > 0$.
\end{lemma}

This result is a consequence of a recent breakthrough arithmetic
progressions in cap sets \cite{e16,clp16} combined with a conditional
result on the Erd\"{o}s-Szemeredi sunflower conjecture \cite{asu13}.
The results of \cite{bccgu16} do imply that Cohn and Umans' strong
uniquely solvable puzzle approach cannot achieve the ideal $\omega =
2$.  However, we are unaware of a concrete lower bound on $\epsilon$
implies by this result.  This means there is a still a substantial gap
in our understanding between what has been acheived by the refinements
of LeGall, Williams, and Stothers, and the impossibility of showing
$\omega = 2$ using the Cohn and Umans' approach.



\section{Checking for Strong USPs}
\label{sec:check}

\subsection{3DM to 3SAT}
\label{subsec:3sat}

\section{Heuristics for Strong USPs}
\label{sec:heuristic}

\section{Searching for Strong USPs}
\label{sec:search}

\section{Results}
\label{sec:results}

\subsection{Strong USPs Found}
\label{subsec:usps_found}

\subsection{Algorithm Performance}
\label{subsec:performance}

\section{Conclusions}
\label{sec:conclusion}



\bibliographystyle{customurlbst/alphaurlpp} \bibliography{references}

\appendix


\end{document}
