\documentclass[11pt]{article}
\usepackage{fullpage}
\usepackage{rpmacros}
\RequirePackage[colorlinks=true]{hyperref}
\hypersetup{
  linkcolor=[rgb]{0,0,0.4},
  citecolor=[rgb]{0, 0.4, 0},
  urlcolor=[rgb]{0.6, 0, 0}
}
\usepackage{mathpazo}
\usepackage{bbm}
\usepackage{todonotes}
\usepackage{lipsum}
\usepackage{setspace}
\usepackage{mdframed}
\usepackage{multicol}
\usepackage{graphicx}
\usepackage{comment}
\usepackage{tikz}
\usetikzlibrary{decorations.pathreplacing,backgrounds}

\usepackage[font=footnotesize]{caption}

\usepackage{amsthm}
\usepackage{thmtools,thm-restate}

\numberwithin{equation}{section}
\declaretheoremstyle[bodyfont=\it,qed=\qedsymbol]{noproofstyle} 

\declaretheorem[numberlike=equation]{axiom}

\declaretheorem[numberlike=equation]{observation}
\declaretheorem[numberlike=equation,style=noproofstyle,name=Observation]{observationwp}
\declaretheorem[name=Observation,numbered=no]{observation*}

\declaretheorem[numberlike=equation]{fact}
\declaretheorem[numberlike=equation]{subclaim}
\declaretheorem[numberlike=equation]{problem}

\declaretheorem[numberlike=equation]{theorem}
\declaretheorem[numberlike=equation,style=noproofstyle,name=Theorem]{theoremwp}
\declaretheorem[name=Theorem,numbered=no]{theorem*}

\declaretheorem[numberlike=equation]{lemma}
\declaretheorem[name=Lemma,numbered=no]{lemma*}
\declaretheorem[numberlike=equation,style=noproofstyle,name=Lemma]{lemmawp}

\declaretheorem[numberlike=equation]{corollary}
\declaretheorem[name=Corollary,numbered=no]{corollary*}
\declaretheorem[numberlike=equation,style=noproofstyle,name=Corollary]{corollarywp}

\declaretheorem[numberlike=equation]{proposition}
\declaretheorem[name=Proposition,numbered=no]{proposition*}
\declaretheorem[numberlike=equation,style=noproofstyle,name=Proposition]{propositionwp}

\declaretheorem[numberlike=equation]{claim}
\declaretheorem[name=Claim,numbered=no]{claim*}
\declaretheorem[numberlike=equation,style=noproofstyle,name=Claim]{claimwp}

\declaretheorem[numberlike=equation]{conjecture}
\declaretheorem[name=Conjecture,numbered=no]{conjecture*}

\declaretheorem[numberlike=equation]{question}
\declaretheorem[name=Question,numbered=no]{question*}

\declaretheorem[name=Open Problem]{openproblem}
\declaretheorem[name=Exercise]{exercise}

\declaretheoremstyle[bodyfont=\it,qed=$\lozenge$]{defstyle} 

\declaretheorem[numberlike=equation,style=defstyle]{definition}
\declaretheorem[unnumbered,name=Definition,style=defstyle]{definition*}

\declaretheorem[numberlike=equation,style=defstyle]{example}
\declaretheorem[unnumbered,name=Example,style=defstyle]{example*}

\declaretheorem[numberlike=equation,style=defstyle]{notation}
\declaretheorem[unnumbered,name=Notation=defstyle]{notation*}

\declaretheorem[numberlike=equation,style=defstyle]{construction}
\declaretheorem[unnumbered,name=Construction,style=defstyle]{construction*}

\declaretheorem[numberlike=equation,style=defstyle]{remark}
\declaretheorem[unnumbered,name=Remark,style=defstyle]{remark*}

\renewcommand{\subsectionautorefname}{Section}
\renewcommand{\sectionautorefname}{Section}
\newcommand{\algorithmautorefname}{Algorithm}



%%% Local Variables: 
%%% mode: latex
%%% TeX-master: "main"
%%% End: 

\usepackage{nth}
\usepackage{intcalc}
\usepackage{etoolbox}
\usepackage{xstring}
\hypersetup{
%breaklinks=true % this is to handle linbreaks in url if latex+dvips+ps2pdf used
                % Thanks to Ben Lee Volk for this fix
}

\usepackage{ifpdf}
\ifpdf
\else
\usepackage[quadpoints=false]{hypdvips}
\fi

\newcommand{\mfbiberr}[1]{\PackageWarning{miforbes-bibtex}{#1}}

\newcommand{\cSTOC}[1]{\nth{\intcalcSub{#1}{1968}}\ Annual\ ACM\ Symposium\ on\ Theory\ of\ Computing\ (STOC\ #1)}
\newcommand{\cFSTTCS}[1]{\nth{\intcalcSub{#1}{1980}}\ International\ Conference\ on\ Foundations\ of\ Software\ Technology\ and\ Theoretical\ Computer\ Science\ (FSTTCS\ #1)}
\newcommand{\cCCC}[1]{\ifnumcomp{#1}{<}{1996}{%1985-1995
\nth{\intcalcSub{#1}{1985}}\ Annual\ Structure\ in\ Complexity\ Theory\ Conference\ (Structures\ #1)}%
{\ifnumcomp{#1}{<}{2015}{%1996-2014
\nth{\intcalcSub{#1}{1985}}\ Annual\ IEEE\ Conference\ on\ Computational\ Complexity (CCC\ #1)}%
{%2015-
\nth{\intcalcSub{#1}{1985}}\ Annual\ Computational\ Complexity\ Conference\ (CCC\ #1)}}}
\newcommand{\cFOCS}[1]{\ifnumcomp{#1}{<}{1966}{%1960-1965
\nth{\intcalcSub{#1}{1959}}\ Annual\ Symposium\ on\ Switching\ Circuit\ Theory\ and\ Logical\ Design (SWCT #1)}%
{\ifnumcomp{#1}{<}{1975}{%1966-1974
\nth{\intcalcSub{#1}{1959}}\ Annual\ Symposium\ on\ Switching\ and\ Automata\ Theory\ (SWAT #1)}%
{%1975--
\nth{\intcalcSub{#1}{1959}}\ Annual\ IEEE\ Symposium\ on\ Foundations\ of\ Computer\ Science\ (FOCS\ #1)}}}
\newcommand{\cRANDOM}[1]{\nth{\intcalcSub{#1}{1996}}\ International\ Workshop\ on\ Randomization\ and\ Computation\ (RANDOM\ #1)}
\newcommand{\cISSAC}[1]{#1\ International\ Symposium\ on\ Symbolic\ and\ Algebraic\ Computation\ (ISSAC\ #1)}
\newcommand{\cICALP}[1]{\nth{\intcalcSub{#1}{1973}}\ International\ Colloquium\ on\ Automata,\ Languages and\ Programming\ (ICALP\ #1)}
\newcommand{\cCOLT}[1]{\nth{\intcalcSub{#1}{1987}}\ Annual\ Conference\ on\ Computational\ Learning\ Theory\ (COLT\ #1)}
\newcommand{\cCSR}[1]{\nth{\intcalcSub{#1}{2005}}\ International\ Computer\ Science\ Symposium\ in\ Russia\ (CSR\ #1)}
\newcommand{\cMFCS}[1]{\nth{\intcalcSub{#1}{1975}}\ Internationl\ Symposium\ on\ the\ Mathematical\ Foundations\ of\ Computer\ Science\ (MFCS\ #1)}
\newcommand{\cEUROSAM}[1]{International\ Symposium\ on\ Symbolic\ and\ Algebraic\ Computation\ (EUROSAM\ #1)}
\newcommand{\cSODA}[1]{\nth{\intcalcSub{#1}{1989}}\ Annual\ ACM-SIAM\ Symposium\ on\ Discrete\ Algorithms\ (SODA\ #1)}
\newcommand{\cSTACS}[1]{\nth{\intcalcSub{#1}{1983}}\ Symposium\ on\ Theoretical\ Aspects\ of\ Computer\ Science\ (STACS\ #1)}



\newcommand{\pSTOC}[1]{Preliminary\ version\ in\ the\ \emph{\cSTOC{#1}}}
\newcommand{\pFSTTCS}[1]{Preliminary\ version\ in\ the\ \emph{\cFSTTCS{#1}}}
\newcommand{\pCCC}[1]{Preliminary\ version\ in\ the\ \emph{\cCCC{#1}}}
\newcommand{\pFOCS}[1]{Preliminary\ version\ in\ the\ \emph{\cFOCS{#1}}}
\newcommand{\pRANDOM}[1]{Preliminary\ version\ in\ the\ \emph{\cRANDOM{#1}}}
\newcommand{\pISSAC}[1]{Preliminary\ version\ in\ the\ \emph{\cISSAC{#1}}}
\newcommand{\pICALP}[1]{Preliminary\ version\ in\ the\ \emph{\cICALP{#1}}}
\newcommand{\pCOLT}[1]{Preliminary\ version\ in\ the\ \emph{\cCOLT{#1}}}
\newcommand{\pCSR}[1]{Preliminary\ version\ in\ the\ \emph{\cCSR{#1}}}
\newcommand{\pMFCS}[1]{Preliminary\ version\ in\ the\ \emph{\cMFCS{#1}}}
\newcommand{\pEUROSAM}[1]{Preliminary\ version\ in\ the\ \emph{\cEUROSAM{#1}}}
\newcommand{\pSODA}[1]{Preliminary\ version\ in\ the\ \emph{\cSODA{#1}}}
\newcommand{\pSTACS}[1]{Preliminary\ version\ in\ the\ \emph{\cSTACS{#1}}}


\newcommand{\STOC}[1]{Proceedings\ of\ the\ \cSTOC{#1}}
\newcommand{\FSTTCS}[1]{Proceedings\ of\ the\ \cFSTTCS{#1}}
\newcommand{\CCC}[1]{Proceedings\ of\ the\ \cCCC{#1}}
\newcommand{\FOCS}[1]{Proceedings\ of\ the\ \cFOCS{#1}}
\newcommand{\RANDOM}[1]{Proceedings\ of\ the\ \cRANDOM{#1}}
\newcommand{\ISSAC}[1]{Proceedings\ of\ the\ \cISSAC{#1}}
\newcommand{\ICALP}[1]{Proceedings\ of\ the\ \cICALP{#1}}
\newcommand{\COLT}[1]{Proceedings\ of\ the\ \cCOLT{#1}}
\newcommand{\CSR}[1]{Proceedings\ of\ the\ \cCSR{#1}}
\newcommand{\MFCS}[1]{Proceedings\ of\ the\ \cMFCS{#1}}
\newcommand{\EUROSAM}[1]{Proceedings\ of\ the\ \cEUROSAM{#1}}
\newcommand{\SODA}[1]{Proceedings\ of\ the\ \cSODA{#1}}
\newcommand{\STACS}[1]{Proceedings\ of\ the\ \cSTACS{#1}}

\newcommand{\arXiv}[1]{\texttt{\href{http://arxiv.org/abs/#1}{arXiv:#1}}}
\newcommand{\farXiv}[1]{Full\ version\ at\ \arXiv{#1}}
\newcommand{\parXiv}[1]{Preliminary\ version\ at\ \arXiv{#1}}

\newcommand{\ECCCurl}[2]{http://eccc.hpi-web.de/report/\ifnumcomp{#1}{>}{93}{19}{20}#1/#2/}
\newcommand{\cECCC}[2]{\href{\ECCCurl{#1}{#2}}{Electronic\ Colloquium\ on\ Computational\ Complexity\ (ECCC),\ Technical\ Report\ TR#1-#2}}
\newcommand{\shortECCC}[2]{\texttt{\href{http://eccc.hpi-web.de/report/\ifnumcomp{#1}{>}{93}{19}{20}#1/#2/}{eccc:TR#1-#2}}}
\newcommand{\ECCC}{Electronic\ Colloquium\ on\ Computational\ Complexity\ (ECCC)}
\newcommand{\fECCC}[2]{Full\ version\ in\ the\ \cECCC{#1}{#2}}
\newcommand{\pECCC}[2]{Preliminary\ version\ in\ the\ \cECCC{#1}{#2}}

\newcommand{\parseECCC}[1]{% Takes a string of the form TRxx/xxx or
%                          % TRxx-xxx and returns short ECCC link
\StrSubstitute{#1}{TR}{}[\tmpstring]%
\IfSubStr{\tmpstring}{/}{ %assuming string is of the form TRxx/xxx
\StrBefore{\tmpstring}{/}[\ecccyear]%
\StrBehind{\tmpstring}{/}[\ecccreport]%
}{% assuming string is of the form TRxx-xxx
\StrBefore{\tmpstring}{-}[\ecccyear]%
\StrBehind{\tmpstring}{-}[\ecccreport]%
}%
\shortECCC{\ecccyear}{\ecccreport}}


\usepackage{algorithmicx}
\usepackage{algorithm} % http://ctan.org/pkg/algorithms
\usepackage[noend]{algpseudocode}
\renewcommand{\algorithmicensure}{\textbf{Output:}}
\renewcommand{\algorithmicrequire}{\textbf{Input:}}


\algrenewcommand\algorithmicindent{1.0em}%

\newcommand{\RPnote}[1]{\textcolor{BrickRed}{RP: #1}}
\newcommand{\Mattnote}[1]{\textcolor{OliveGreen}{MWA: #1}}
\newcommand{\BLnote}[1]{\textcolor{Blue}{BLV: #1}}
\newcommand{\Anote}[1]{\textcolor{Plum}{A: #1}}
\newcommand{\MFnote}[1]{\textcolor{DarkOrchid}{MF: #1}}

\newcommand*\samethanks[1][\value{footnote}]{\footnotemark[#1]}
\newcommand\sse{\subseteq}
\newcommand\Sym[1]{\ensuremath{\mathrm{Sym}_{#1}}}
\newcommand\condset[2]{\set{#1 \;|\; #2}}
\renewcommand\NP{\ensuremath{\mathsf{NP}}}
\newcommand\coNP{\ensuremath{\mathsf{coNP}}}

%\onehalfspacing
\date{}

\title{Matrix Multiplication: Finding Strong Uniquely-Solvable Puzzles
{\IfFileExists{./sha.tex}{\\\small SHA: \input{sha}}{}}}
\author{
Matthew Anderson\thanks{Department of Computer Science, Union College, Schenectady, New York, USA, E-mails: \texttt{andersm2@union.edu, jiz@union.edu, xua@union.edu}}%
\and%
Zongliang Ji\samethanks[1]
\and%
Anthony Yang Xu\samethanks[1]
}
\begin{document}
\maketitle

\begin{abstract}
Cohn and Umans proposed a framework for developing fast matrix
multiplication algorithms based on the embedding computation in
certain groups \cite{cu03}.  Subsequent work with Kleinberg and
Szegedy connected this to the search for combinatorial objects called
strong uniquely solvable puzzles (SUSP) \cite{cksu05}.  We begin a
systematic computer-aided search for these objects.  We develop and
implement algorithms and heuristics to verify that puzzles are SUSP
and to search for large ones.  We produce tight bounds on the maximum
size of puzzles for width $k < 6$, and construct puzzles of small
width that are larger than previous work.  Although our work only
deals with puzzles of small-constant width and does not produce a new
faster algorithm, it gives evidence that there exist families of SUSP
that imply matrix multiplication algorithms that are more efficient
than those currently known.
\end{abstract}

\thispagestyle{empty}
\newpage
\pagenumbering{arabic}


\section{Introduction}
\label{sec:intro}

% Context

An optimal algorithm for matrix multiplication remains elusive
despite substantial effort.  We focus on the square variant of the
matrix multiplication problem, that is, given two $n$-by-$n$ matrices
$A$ and $B$ over a field $\F$, the goal is to compute the matrix
product $C = A \times B$.  The question that arised is: How many field
operations are required to compute $C$.  The na\"{i}ve algorithm,
based on the mathematical definition of matrix product, runs in time
$O(n^3)$, and for a time it was thought to be the optimal algorithm.
It was surprising when Strassen showed that matrix multiplication can
be done in time $O(n^{2.808})$ \cite{str69} using a
divide-and-conquer approach.  A long sequence of work concluding with
Coppersmith and Winograd's laser method reduced the running time
to $O(2^{2.376})$ \cite{pan78,b79,sch81,cw82,str86,cw87}. Recent
computer-aided refinements of Coppersmith and Winograd's work done by
Stothers, Vassilevska-Williams, and Le Gall further reduced the
exponent to $\omega \le 2.3728639$ \cite{sto10,vas11,leg14}.

% Other lower bounds?
% Tensor rank approach
% - Lower bounds
% - Impossibility

% Motivation

Cohn and Umans \cite{cu03} introduced a framework for developing
faster algorithms for matrix multiplication by reducing this question
to a search for groups with subsets that satisfy a certain algebraic
property called the \emph{triple-product property} that allows matrix
multiplication to be embedded in the group algebra.  Subsequent work
\cite{cksu05} elaborated on this idea and developed the notion of
combinatorial objects called \emph{strong uniquely solvable puzzles}
(SUSP) whose existence would also imply faster matrix multiplication
algorithms.

A \emph{width}-$k$ puzzle $P$ is a subset of $U_k = \set{0,1,2}^k$,
and the cardinality of $P$ is the puzzle's \emph{size}.  Each element
of $P$ is called a \emph{row} of $P$, and each row consists of three
\emph{subrows} that are elements of $\set{0,*}^k$, $\set{1,*}^k$,
$\set{2,*}^k$ respectively.  Informally, a puzzle $P$ is a
\emph{uniquely solvable puzzle} if there is no way to permute the
subrows of $P$ without overlap to form a distinct puzzle $P'$.  A
uniquely solvable puzzle is \emph{strong} if a tighter condition for
non-overlapping is applied. For a fixed width $k$, the larger the size
of a puzzle $P$, the faster matrix multiplication algorithm it implies
if $P$ is a SUSP \cite[Corollary 3.6]{cksu05}.  In fact Cohn et
al.~show that there exist an infinite family of SUSP that achieves $w <
2.48$ \cite[Proposition 3.8]{cksu05}.

% Approach

We follow Cohn and Umans' program simulateneously from theoretical and
experimental perspectives.  We (i) \textbf{explore} properties of SUSP, (ii)
develop \textbf{verification} algorithms for determining whether a
puzzle is a SUSP, (iii) develop \textbf{search} algorithms for searching
for large SUSP, and (iv) \textbf{implement} these algorithms in desktop and
high performance computing settings.  From the computational
complexity perspective the algorithms we develop to verify and search
for SUSP are not efficient as they run in exponential or doubly
exponential time in the natural parameters.  However, as the goal is
to find a sufficiently large SUSP which in turn produces a fast matrix
multiplication algorithm, the inefficiency of our algorithms does not
directly impact the efficiency of the matrix multiplication
algorithms.  Rather, it indirectly impacts the efficiency by limiting
the search space of puzzles that we can practically examine.

% Results

In addition to the various theoretical results, we have experimental
results that bound the size of the largest SUSP for small width
puzzles.  For small-constant width ($k \le 5$) the bounds are tight,
though the tightness of the results decreases as width increases.
Lower bounds on size are witnessed by examples of SUSP we found via
search and constructions that compose SUSP of small width into SUSP of
larger width while maintaining the relative size of the SUSP.  Upper
bounds are determined computationally by evaluating admissible
heuristics on the initial levels of the puzzle search tree.  Although
our current experimental results do not beat Proposition 3.8 of
\cite{cksu05} for unbounded $k$, it gives evidence that there exist
families of SUSP that imply matrix multiplication algorithms that are
more efficient than those currently known.

There are a number of negative results in this area.  First, there are
some lower bounds on the time complexity of matrix multiplication.
Na\"{i}vely, the dimensions of the output matrix $C$ imply that the
problem requires at least $\Omega(n^2)$ time.  Slightly better lower
bounds are also known for specialized models of computation, like
bounded-depth arithmetic circuits \cite{XXX}.

There are also lower bounds known for a variety of algorithmic
approaches for matrix multiplication.  Ambainis et al.~showed that the
laser method, the approach of Coppersmith-Winograd and subsequent
refinements, cannot alone achieve an algorithm with $\omega \le
2.3078$ running time \cite{afl14}.  A recent breakthrough arithmetic
progressions in cap sets \cite{e16,clp16} combined with a conditional
result on the Erd\"{o}s-Szemeredi sunflower conjecture \cite{asu13}.
The results of \cite{bccgu16} do imply that Cohn et al.'s strong
uniquely solvable puzzle approach cannot achieve the $\omega = 2 +
\epsilon$ for some $\epsilon > 0$.  Subsequent work has extended and
generalized this barrier \cite{avw18a,avw18b} to a larger
class of algorithmic techniques.  Despite this, we are unaware of a
concrete lower bound on $\epsilon$ implied by these negative results.
There remains a substantial gap in our understanding
between what has been acheived by the positive refinements of LeGall, Williams,
and Stothers, and the impossibility of showing $\omega = 2$ using the
uniquely solvable puzzles approach.

% Organization
\paragraph{Organization}
We begin with formal definitions and properties of puzzles and strong
uniquely solvable puzzles in \autoref{sec:prelim}.  In
\autoref{sec:verify} we discuss our algorithms for verifying that a
puzzle is a SUSP.  In \autoref{sec:heuristic} we describe fast,
but incomplete, heuristics for verifying SUSP.
\autoref{sec:search} explains our search algorithms.
\autoref{sec:results} discuss our implementation, experiments, and
the experimental results.


\section{Preliminaries}
\label{sec:prelim}

\newcommand\ordset[1]{\ensuremath{[#1]}}

We use $\ordset{n}$ to denote the set $\set{0,1,2,\ldots, n-1}$.  For
a set $Q$, $\Sym{Q}$ denotes the symmetric group on the elements of
$Q$.  We use $1$ to denote the identity permutate in
$\Sym{Q}$. \cite{cksu05} introduced the idea of a \emph{puzzle}.

\begin{definition}[Puzzle]
  For $s, k \in \Natural$, an $(s,k)$-\emph{puzzle} is a
  subset $P \sse U_k = \ordset{3}^k$ with $|P| = s$.
\end{definition}

We say that an $(s,k)$-puzzle has $s$ rows and $k$ columns.  The
columns are inherent ordered and indexed by $\ordset{k}$.  The rows are not
inherently ordered, however, it is often convienent to assume that the
rows are arbitrarily ordered and indexed by $\ordset{s}$.

\cite{cksu05} also establish a particular combinatorial property of
such puzzles that can derive groups that matrix multiplication can be
embedded into.  Such puzzles are called \emph{strong} uniquely
solvable puzzles.  However, to give some intuition we first explain a
simpler version of the property called \emph{uniquely solvable
  puzzles}.

\begin{definition}[Uniquely Solvable Puzzle (USP)]
  \label{def:strong-USP}
  ~\\An $(s,k)$-puzzle $P$ is \emph{uniquely solvable} if
  $\forall \pi_0, \pi_1, \pi_2 \in \Sym{P}:$
  \begin{enumerate}
  \item either $\pi_0 = \pi_1 = \pi_2$, or
  \item $\exists r \in P, \exists i \in \ordset{k}$ such that at least two
    of the following hold:
    \begin{enumerate}
    \item $(\pi_0(r))_i = 0$,
    \item $(\pi_1(r))_i = 1$,
    \item $(\pi_2(r))_i = 2$.
    \end{enumerate}
  \end{enumerate}
\end{definition}

Informally, a puzzle is \textbf{not} uniquely solvable if each row of
the puzzle can be broken into zeroes, ones, and twos pieces and then
the rows can be reassembled in a different way so that each new row is
a combination of a zeroes, a ones, and twos piece where there is
exactly one element of $\ordset{3}$ for each column.  Observe that
uniquely solvable puzzles can have at most $2^k$ rows because each
zeroes piece, ones piece, and two piece must be unique, as otherwise
the duplicate pieces can be swapped making the puzzle not uniquely
solvable.  The definition of \emph{strong} uniquely solvable puzzle is
below, it is nearly the same except that it requires that there be a
collision on a column between exactly two pieces, not two or more
pieces like in the original definition.

\begin{definition}[Strong Uniquely Solvable Puzzle]
  ~\\
  An $(s,k)$-puzzle $P$ is \emph{strong uniquely solvable} if
  $\forall \pi_0, \pi_1, \pi_2 \in \Sym{P}:$
  \begin{enumerate}
  \item either $\pi_0 = \pi_1 = \pi_2$, or
  \item $\exists r \in P, \exists i \in \ordset{k}$ such that exactly two
    of the following hold:
    \begin{enumerate}
    \item $(\pi_0(r))_i = 0$,
    \item $(\pi_1(r))_i = 1$,
    \item $(\pi_2(r))_i = 2$.
    \end{enumerate}
  \end{enumerate}

\end{definition}

Observe that the properties of uniquely solvable and strong uniquely
solvable are invariant to the ordering of the rows or columns of a
puzzle.  This fact is used implicitly.  Note that if a puzzle $P$ is a
SUSP then it is also a USP.  Further note that these properties are
invariant to maps between puzzles induced by permutations from
\Sym{\ordset{3}} on the puzzle cell elements.

Cohn et al.~show the following connection between the existence of
strong uniquely solvable puzzles and upper bounds on the exponent of
matrix multiplication $\omega$.
\begin{lemma}[{\cite[Corollary 3.6]{cksu05}}]
  \label{lem:omega}
  If there is a strong uniquely solvable $(s,k)$-puzzle,
  $$\omega \le \min_{m \ge 3, m \in \Natural} \frac{3 \log
    m}{\log(m-1)} - \frac{3 \log s!}{sk \log(m-1)}.$$
\end{lemma}
\noindent In the same article, the authors also demonstrate the
existence of an infinite family of strong uniquely solvable puzzles, for width $k$ divisible by three,
that achieves a non-trivial bound on $\omega$.
\begin{lemma}[{\cite[Proposition 3.8]{cksu05}}]
  There is an infinite family of strong uniquely solvable puzzles that
  achieves $\omega < 2.48$.
\end{lemma}
\noindent Finally, they conjecture that strong uniquely solvable
puzzles provide a root to achieving quadratic time matrix
multiplication.
\begin{conjecture}[{\cite{cksu05}}]
  There exists a family of strong uniquely solvable puzzles that
  implies $\omega = 2$.
\end{conjecture}
\noindent Unfortunately, as mentioned in the introduction, this conjecture was recently shown to be false.
\begin{lemma}[\cite{bccgu16}]
  Strong uniquely solvable puzzles cannot show $\omega < 2 +
  \epsilon$, for some $\epsilon > 0$.
\end{lemma}
That said, there remains hope that the uniquely solvable puzzle
approach could beat the refinements of Coppersmith-Winograd even if it
cannot reach $\omega = 2$.

%% \noindent This result is a consequence of a recent breakthrough
%% arithmetic progressions in cap sets \cite{e16,clp16} combined with a
%% conditional result on the Erd\"{o}s-Szemeredi sunflower conjecture
%% \cite{asu13}.  The results of \cite{bccgu16} do imply that Cohn and
%% Umans' strong uniquely solvable puzzle approach cannot achieve the
%% ideal $\omega = 2$.  However, we are unaware of a concrete lower bound
%% on $\epsilon$ implies by this result.  This means there is a still a
%% substantial gap in our understanding between what has been acheived by
%% the refinements of LeGall, Williams, and Stothers, and the
%% impossibility of showing $\omega = 2$ using the Cohn and Umans'
%% approach.



\section{Verifying Strong USPs}
\label{sec:verify}

The core focus of this article is the verification of strong
uniquely-solvable puzzles.  In particular, we are interested in the
decision problem: Given an $(s,k)$-puzzle $P$, output YES iff $P$ is a
strong uniquely-solvable puzzle.  In this section we discuss the
design of efficient and practical algorithms to solve this problem as
a function of the parameters $s$ and $k$.  As the goal of this work is
to use computers to locate large strong USPs, we also discuss aspects
of our implementation that informed or constrained our designs.  All of
the exact algorithms we develop in this section have exponential
running time in the parameters $s$ and $k$.  However, although we
discuss the asymptotic worst-case running time of our algorithms, this
is not the metric we are truly interested in.  Rather we are
interested in the practical performance of our algorithms and their
capability for locating new large strong USPs.

The algorithm that we ultimately develop and implement in this section
is a hybrid of a number of simpler algorithms and heuristics.  The
primary reason for this is that the algorithms with better asymptotic
and practical performance at larger input lengths have large overhead
at small input lengths.  Although we have two parameters $s$ and $k$
they are not fully independent.  First, $s \le 3^k$ because the
maximum number of rows in a puzzle of width $k$ is $|[3]^k| = 3^k$.
Second, we can eliminate the dependence on $k$ entirely by
transforming an $(s,k)$-puzzle into a 3D matching instance on the
vertex set $[s]^3$.  However, this transformation is not free because
the instance size is now a function of the cube of $s$ rather than
linear in $s$.

\subsection{Brute Force}

The obvious algorithm for verification comes directly from the
definition of strong uniquely solvable puzzles.  Informally, we
consider all ways of permuting the ones and twos pieces relative to
the zeroes pieces and check whether the non-overlapping condition of
\autoref{def:strong-USP} is met.  A formal description of the
algorithm is found \autoref{alg:brute-force}.

\begin{algorithm}
  \caption{: Brute Force}
  \label{alg:brute-force}
\begin{algorithmic}[1]
  \Require{An $(s,k)$-puzzle $P$.}
  \Ensure{YES, if $P$ is a strong USP and NO otherwise.}
  \Function{VerifyBruteForce}{$P$}
  \For{$\pi_1 \in \Sym{P}$}
    \For{$\pi_2 \in \Sym{P}$}
      \If{$\pi_1 \neq 1 \vee \pi_2 \neq 1$}
        \State{$found = false.$}
        \For{$r \in P$}
          \For{$i \in [k]$}
            \If{$\delta_{r_i, 0} + \delta_{(\pi_1(r))_i, 1} + \delta_{(\pi_2(r))_i, 2} = 2$} $found = true$. \EndIf
          \EndFor
        \EndFor
        \If{not $found$} \Return{NO.} \EndIf
      \EndIf
    \EndFor
  \EndFor
  \State{\Return{YES}.}
  \EndFunction
\end{algorithmic}
\end{algorithm}
Note that the $1$ in Line 4 of \autoref{alg:brute-force} denotes the identity permutation in $\Sym{P}$,
and $\delta_{a,b}$ is the Kronecker delta function which is $1$ if $a
= b$ and $0$ otherwise.  Observe that \autoref{alg:brute-force} does
not refer to $\pi_0$ of \autoref{def:strong-USP}.  This is because the
strong USP property is invariant to permutations of the rows and so
$\pi_0$ can be thought of as an arbitrary phase.  Hence we fix $\pi_0 = 1$
to simplify the algorithm.  Seeing that $|\Sym{P}| = s!$, the four
nested loops make the running time of this algorithm easy to
analyze.  The algorithm runs in time $O((s!)^2 \cdot s \cdot k \cdot
\poly(s))$ where the last factor accounts
for the time to perform operations on permutations of $s$ elements.  The dominant term
in the running time is the contribution from iterating over pairs of
permutations, though this term was made a factor $s!$ smaller by not
iterating over $\pi_0$.  Finally, notice that if $P$ is a strong USP,
then the algorithm runs in time $\Theta((s!)^2 \cdot k \cdot
\poly(s))$, and that if $P$ is not a strong USP the algorithm may
terminate early.  This algorithm's poor theoretical and practical
performance made it unusable in our implementation, however, it's
simplicity and direct connection to the definition made its
implementation a valuable sanity check against later more elaborate
algorithms (and served as effective onboarding to the undergraduate
students collaborating on this project).

Although \autoref{alg:brute-force} performs poorly, examining the
structure of a seemingly trivial optimization leads to substantially
more effective algorithms. Consider the following function on triples of rows $a, b,
c$: $$f(a,b,c) = \vee_{i \in [k]} (\delta_{a_i,0} + \delta_{b_i,1} +
\delta_{c_i,2} = 2).$$ We can replace the innermost loop in Lines 6 \&
7 of \autoref{alg:brute-force} with the statement $found = found \vee
f(r, \pi_1(r), \pi_2(r))$.  Observe that $f$ neither depends on $P$,
$r$, nor the permutations, and that \autoref{alg:brute-force} no
longer depends on $k$.  To slightly speed up \autoref{alg:brute-force}
we can precompute and cache $f$ before the algorithm starts and then
look up values as the algorithm runs.  There are two obvious options
to consider we can either precompute $f$ specialized to the rows in
the puzzle $P$, or we can precompute $f$ for all possible rows.  In
former case the time to precompute $f$ is $\Theta(s^3 \cdot k)$ and in
the later case $\Theta(3^{3k} \cdot k)$.  The storage requirements are
$\Theta(s^3)$ and $\Theta(3^{3k})$ bits respectively.  The former is
problematic for large $s$ and later problematic even for small $k$.
Moreover the combined running time for the two options with a single
call to verify a puzzle is $\Theta(s^3 \cdot k + (s!)^2 \cdot
\poly(s))$ and $\Theta(3^{3k} \cdot k + (s!)^2 \cdot \poly(s))$.  In
the former case there is an asymptotic improvement, but in the later
case, the saving of a factor of $k$ is easily offset by the additional
$3^{3k}$ term.  For this reason we rule out the later option and chose
to represent the function $f$ specialized to $f_P$ for a given puzzle
$P$.

\subsection{Verification to 3D Matching}

For verification it turns out to be more useful to work with $f_P$
than with $P$.  It is convienent to think of $f_P$ from several
perspectives: (i) as a function $f_P : P \times P \times P \rightarrow
\set{0, 1}$, (ii) as an order-three tensor $f_p \in \set{0,1}^{P
  \times P \times P}$, and (iii) as the complement of the
characteristic function of hyperedge relations of a tripartite
hypergraph $H_P = \langle P \sqcup P \sqcup P, \bar{f_p}\rangle$ where
the vertex set is the disjoint union of three copies of $P$.  The last
of these, $H_P$, is the most useful.

Let $G = \langle P \sqcup P \sqcup P, E \sse P^3\rangle$ be a
tripartite 3-hypergraph.  We say that $G$ has a \emph{3D matching} iff
there exists a subset $M \sse E$ with $|M| = |P|$ and for all distinct
hyperedges $e_1, e_2 \in M$, $e_1$ and $e_2$ are \emph{vertex
  disjoint}, that is, $(e_1)_i \neq (e_2)_i$ for all $i \in [3]$.  We
say a 3D matching is \emph{nontrival} if it is not the matching
$\condset{(r,r,r)}{r \in P}$.  The existence of 3D matchings in $H_P$
is directly tied to whether $P$ is a SUSP.

\begin{lemma}
  \label{lem:verify-to-3dm}
  An $(s,k)$-puzzle $P$ is a strong USP iff $H_P$ has no nontrivial 3D
  matching.
\end{lemma}

\begin{proof}
  We first argue the reverse direction.  Suppose that $H_p$ has a
  nontrivial 3D matching $M$.  We show that $P$ is not a strong USP by
  using $M$ to construct $\pi_0, \pi_1, \pi_2 \in \Sym{P}$ that
  witness this.  Let $\pi_0$ be the identity permutation.  For each $r
  \in P$, define $\pi_1(r) = q$ where $(r,q,\_) \in M$.  Note that $q$
  is well defined and unique because $M$ is 3D matching and so has vertex
  disjoint edges.  Similarly define $\pi_2(r) = q$ where $(r,\_,q) \in
  M$.  Observe that by construction $M =
  \condset{(\pi_0(r),\pi_1(r),\pi_2(r))}{r \in P}$.  Since $M$ is a
  matching of $H_P$, $M \sse \bar{f_P}$.  Because $M$ is a nontrival
  matching at least one edge in $(a,b,c) \in M$ is has either $a \neq
  b$, $a \neq c$, or $b \neq c$.  This implies, respectively, that as
  constructed $\pi_0 \neq \pi_1$, $\pi_0 \neq \pi_2$, or $\pi_1 \neq
  \pi_2$.  In each case we have determined that $\pi_0$, $\pi_1$, and
  $\pi_2$ are not all identical.  Thus we determined permutations such
  that for all $r \in P$, $f(\pi_0(r), \pi_1(r), \pi_2(r)) = 0$.  This
  violates Condition 2 of \autoref{def:strong-USP}, hence $P$ is not a
  strong USP.

  The forward direction of the lemma is argued symmetrically.  Suppose
  that $P$ is not a strong USP. We show that $H_P$ has a 3D matching.
  For $P$ not to be a strong USP there must exist $\pi_0, \pi_1, \pi_2
  \in \Sym{P}$ not all identical such that Condition 2 of
  \autoref{def:strong-USP} fails.  Define $e(r) =
  (\pi_0(r),\pi_1(r),\pi_2(r))$ and $M = \condset{e(r)}{r \in P}$.
  Since Condition 2 fails, we have that $f_P(e(r)) = false$ for all $r
  \in P$.  This means that for all $r \in P$, $e(r) \in \bar{f_P}$ and
  hence $M \sse \bar{f_P}$.  Since $\pi_0$ is a permutation $|M| =
  |P|$.  Finally, observe that $M$ is nontrivial because not all of
  the permutations are identical and there must be some $r \in P$ with
  $e(r)$ having non identical coordinates.  Thus $M$ is a nontrivial
  3D matching of $H_p$.
\end{proof}

Although 3D matching is an \NP-complete problem \cite{karp72},
\autoref{lem:verify-to-3dm} does not immediately imply that
verification of strong USP is \coNP-complete because $H_P$ is not an
arbitrary hypergraph.  (That verification is in \coNP{} is immediate
from \autoref{def:strong-USP}.)  Indeed, it seems difficult to define a
puzzle $P$ that allows hyperedges in $H_P$ to be included
independently.  It remains open whether verification is
\coNP-complete, but our implementation and experimental data suggests
that it is likely not to be the case.

\autoref{lem:verify-to-3dm} implies that to verify $P$ is a strong USP
it suffices to determine whether $H_P$ has a 3D matching.  In the
subsequent sections we examine algorithms for the later problem.  We
can also view \autoref{alg:brute-force} as algorithm for solving 3D
matching.  Since we believe that verification is not \coNP-complete we
use properties of strong USP and puzzles to provide additional
structure that algorithms can take advantage of.

XXX - Discuss simplifyTDM.

\subsection{Bidirectional Search}

The realization that verification of strong USP is a variant of 3D
matching quickly led to a linear exponential time algorithm that was more practical.  The reduction allows us to replace the permutations
from $\Sym{P}$ with subsets of $P$ and effectively reduce the cost of
the outer loops of \autoref{alg:brute-force} from $s! =
\Theta(2^{s\log s})$ to $2^s$.  \autoref{alg:bi} describes a recursive
bidirectional strong USP verification algorithm that using the 3D matching instance.

\begin{algorithm}
  \caption{: Bidirectional}
  \label{alg:bi}
\begin{algorithmic}[1]
  \Require{An $(s,k)$-puzzle $P$.}
  \Ensure{YES, if $P$ is a strong USP and NO otherwise.}
  \Function{VerifyBidirectional}{$P$}
  \State{Let $T = \emptyset$.}
  \State{Construct 3DM instance $H_P$.}
  \Function{SearchHalf}{$\ell, Q,\ell_Q, R,\ell_R, \delta, t$}
  \If{$\ell = t$}
    \If{$\delta = 1$} \Comment{Forward Base Case}
      \State{Insert $(Q,R)$ into $T$.}
      \State{\Return{$false$}.}
    \Else \Comment{Reverse Base Case}
      \If{$(P-Q, P-R) \in T$} \State{\Return{$true$}.} \Else \State{\Return{$false$}.} \EndIf
    \EndIf
  \EndIf
  \State{$result = false$.} \Comment{Recursive Case}
  \For{$\ell'_Q = \ell_Q + 1$ to $s$}
    \For{$\ell'_R = \ell_R + 1$ to $s$}
      \If{$(p_\ell, p_{\ell'_Q}, p_{\ell'_R}) \in H_P$}
        \State{$result = result~\vee$ \textsc{SearchHalf}$(\ell + \delta, Q \cup \set{p_{\ell'_Q}}, \ell'_Q, R \cup \set{p_{\ell'_R}}, \ell'_R, \delta, t)$.}
      \EndIf
    \EndFor
  \EndFor
  \State{\Return{$result$.}}
  \EndFunction

  \State{\textsc{SearchHalf}$(1,\emptyset, 0, \emptyset, 0, 1, \lfloor s / 2 \rfloor + 1)$.}
  \State{\Return{\textsc{SearchHalf}$(s, \emptyset, 0, \emptyset, 0, -1, \lfloor s/2 \rfloor)$}.}
  \EndFunction
\end{algorithmic}
\end{algorithm}

\autoref{alg:bi} consists of two phases. Let $t = \lfloor s/2 \rfloor$. The
first phase determines all possible sets $Q,R \sse P$ with $|Q| = |R|
= t$ such that there is 3D matching $M_1$ of $H_P$ when restricted to
the vertices $\set{p_1,p_2, \ldots, p_t} \sqcup Q \sqcup R$.  The sets
$Q,R$ satisfying the requirement are stored in a table $T$ during the first
phase on Line 6.  The second phase determines all possible sets $Q,R
\sse P$ with $|Q| = |R| = s - t$ such that there is a 3D matching
$M_2$ of $H_P$ when restricted to the vertices
$\set{p_{t+1},p_{t+2},\ldots,p_s} \sqcup Q \sqcup R$.  For each pair
$(Q,R)$ the algorithm determines in the second phase it checks whether
$(P - Q, P - R)$ was inserted into $T$ during the first phase.  If the
pair is present it means that there is a 3D matching of $H_P$ which is
$M = M_1 \cup M_2$.  This works because $M_1$ and $M_2$ are partial 3D
matchings and vertex disjoint.  This is because $M_1$ is a 3D matching
on $\set{p_1,\ldots,p_t}$ and $M_2$ is a 3D matching on
$\set{p_{t+1},\ldots p_s}$, and because in Line 9 of \autoref{alg:bi}
the lookup is performed on the complementary sets.  The second phase
returns whether a complete matching could be found, and hence by
\autoref{lem:verify-to-3dm} whether $P$ is a strong USP.  (Note that
the first phase always returns $false$.)

The running time of this algorithm is dominated by the number of
pairs of sets $(Q,R)$ it examines.  Observe that rows of $P$ are
considered in order in Lines 14 \& 15.  Further, the algorithm tracks
the index of last elements added to $Q$ and $R$ in $\ell_Q$ and
$\ell_R$ respectively.  The algorithm only adds new elements to $Q$ or
$R$ that have higher indexes than ones previously added.  Altogether
this implies that each pair of sets $(Q,R)$ is only considered at most
once during a phase.  Since $Q, R \sse P$, there are at most $2^s
\cdot 2^s$ pairs $(Q,R)$.  This means that \textsc{SearchHalf} is
called at most $4^s$ times during each phase.  Hence the running time
of the algorithm is $O(4^s \cdot s^2 \cdot \poly(s) + T_{3DM}(s,k))$
where $s^2$ factor comes from the inner loops and $T_{3DM}(s,k)$
accounts for the time to construct $H_P$.  The memory requirements of
\autoref{alg:bi} are similarly high -- the first phase uses $O(4^s
\cdot s)$ to store $T$.

Note that \autoref{alg:bi} does not early terminate on $P$ that are strong
USP, because it must search through all pairs before determining that
none can be found.  The algorithm could be modified to allow early
termination when $P$ is not a strong USP by causing the second phase
of search to immediately return in Line 17 once the first 3D matching
witness has been located.  However, this still requires the first
phase to run to completion.  A remedy for this would be to run both
phases in parallel and have them check against each other.  This would
substantially complicate the implementation, and not substantially
improve the practical running time for puzzles that are strong USP.

In the implementation of \autoref{alg:bi} we represented the sets
$Q,R$ using bit sets encoded into a single 64-bit long.  This
permitted it to represent these sets for $s \le 32$.  We implemented
$T$ using a hash table (\texttt{map}s from the C++ standard template library).
These choices made the basic data structure operations in
implementation effectively constant time, and have low memory overhead
beyond the contents of $T$.

Note that more advanced techniques like those of Bj\"{o}rklund et
al.~can acheive a better asymptotic time of $O(2^s\poly(s))$
\cite{b2010}.  We chose not to implement their algorithm, because we
judged that it would not substantially increase the domain of which
verification was possible.  We instead took a different approach.

\newpage
\subsection{3DM to 3SAT}
\label{subsec:sat}

By \autoref{lem:verify-to-3dm}, one can
determine whether a puzzle $P$ is a strong USP by contructing a
graph $H_P$ from $P$ and deciding whether it has a non-trivial 3D matching.  It well-known that 3D matching is \NP{}-complete \cite{XXX}, and it can be efficiently reduced to
other \NP{}-complete problems. Here we reduce our 3D
matching problem to the
CNF satisfability (SAT) problem and then use a state-of-the-art
SAT solver (MapleCOMPSPS \cite{XXX}) to answer the reduced problem. To perform the
reduction, we convert the graph $H_P$ that is obtained from the
puzzle $P$ into a conjunctive normal form (CNF) formula, $\Psi_P$, a depth-2 formula that is the AND of ORs of Boolean literals.  We input the CNF
formula $\Psi_P$ into a SAT solver to determine whether the formula is
satisfiable.  We construct $\Psi_P$ so that \autoref{lem:verify-to-3dm} implies that $\Psi_P$
is satisfiable iff $H_P$ has a non-trivial \emph{3D matching}.

Let $H_P = \langle V, E \rangle$ be the 3D matching instance
associated with the puzzle $P$.  Our goal is to construct a
non-trivial 3D matching $M$ that is a subset of the edges of $E$ and
is vertex disjoint.  We use $M_{u,v,w}$ to denote a variable with
values in $\set{0,1}$ that indicates whether the edge $(u,v,w) \in P^3$
is present in the matching.

We describe the clauses of the CNF formula $\Phi_P$ whose
satisfiability captures the existence of a 3D matching in $H_P$.
First, we have clauses that prevents non-edges from being included:
\begin{equation}
  \Phi_P^{\text{non-edge}} = \bigwedge_{(u,v,w) \in \overline{E}} \neg M_{u,v,w}.
\end{equation}
Note that, alternatively, we could just eliminate these variables by
replacing them by constant false in the subsequent formula. Second, we
add clauses that require that every vertex in $H_P$ is matched with some
hyperedge:
\begin{equation}
  \Phi_P^{\ge 1} = 
  \left(\bigwedge_{u \in P}\vee_{v,w \in P} ~M_{u,v,w}\right)
  \wedge \left(\bigwedge_{v \in P}\vee_{u,w \in P} ~M_{u,v,w}\right)
  \wedge \left(\bigwedge_{w \in P}\vee_{u,v \in P} ~M_{u,v,w}\right).
\end{equation}
Note that three types of clauses above account for the three disjoint
copies of $P$ in the vertices of $H_P$.  Third, we require that each
vertex can only be matched one a single hyperedge and so have clauses
that exclude matching hyperedges that overlap non-trivially (on one or two
coordinates):
\begin{equation}
  \Phi_P^{\le 1} =
    \bigwedge_{(u,v,w), (u',v',w') \in P^3} (u = u' \vee v = v' \vee w = w') \wedge  (u,v,w \neq u',v',w') \Rightarrow \neg M_{u,v,w} \vee \neg M_{u',v',w'}.
    %% &\left[\bigwedge_{u \in P} \bigwedge_{v, w \in P} \left(\bigwedge_{(v', w') \in P^2 \backslash \set{(v,w)}} \overline{M_{u,v,w}} \vee \overline{M_{u,v',w'}}\right)\right] \\
    %% \wedge &\left[\bigwedge_{u \in P} \bigwedge_{v, w \in P} \left(\bigwedge_{(v', w') \in P^2 \backslash \set{(v,w)}} \overline{M_{u,v,w}} \vee \overline{M_{u,v',w'}}\right)\right] \\
    %% \wedge &\left[\bigwedge_{u \in P} \bigwedge_{v, w \in P} \left(\bigwedge_{(v', w') \in P^2 \backslash \set{(v,w)}} \overline{M_{u,v,w}} \vee \overline{M_{u,v',w'}}\right)\right].
\end{equation}
Fourth, we exclude the trivial 3D matching by requiring that at least
one of diagonal hyperedges not be used:
\begin{equation}
  \Phi_P^{\text{non-trivial}} = \bigvee_{u \in P} \neg M_{u,u,u}.
\end{equation}
Finally, we AND these into the overall formula:
\begin{equation}
  \Phi_P = \Phi_P^{\text{non-edge}} \wedge \Phi_P^{\le 1} \wedge \Phi_P^{\ge 1} \wedge \Phi_P^{\text{non-trivial}}.
\end{equation}
The size of the CNF formula $\Phi_P$ dominated by the size of
$\Phi_P^{\le 1}$ which is $\Theta(s^5)$ size.  This is relatively
large compared to the $\Theta(s^3)$ size required to represent the
hyperedges of the 3DM instance.

To avoid the polynomial growth in instance size we decided to use a
more concise matching representation.  Rather than having variables
$M_{u,v,w}$ for each hyperedge possible in the matching, we uses sets
of variables to indicate which vertices in the second and third part
are matched with each vertex in the first part.  In particular we have
Boolean variables $M_{u,v}^1$ and $M_{u,w}^2$ for all $u, v, w \in P$,
and these variable map to assignments in the original scheme in the
following way: $M_{u,v}^1 \wedge M_{u,w}^2 \Leftrightarrow M_{u,v,w}$.

We can concisely rewrite our CNF formula for 3D matching using these
new variables.  First, we have clauses that prevents non-edges from
being in the matching:
\begin{equation}
  \Psi_P^{\text{non-edge}} = \bigwedge_{(u,v,w) \in \overline{E}} (\neg M_{u,v}^1 \vee \neg M_{u,w}^2).
\end{equation}
Next, we add clauses require that every vertex in $H_P$ is matched with some
hyperedge:
\begin{equation}
  \Psi_P^{\ge 1} =
  \left(\bigwedge_{u \in P} (\vee_{v\in P} ~M_{u,v}^1) \wedge (\vee_{w \in P} ~M_{u,w}^2)\right)
  \wedge \left(\bigwedge_{v \in P} (\vee_{u \in P} ~M_{u,v}^1)\right)
  \wedge \left(\bigwedge_{w \in P} (\vee_{u \in P} ~M_{u,w}^2)\right).
\end{equation}
Note that these clauses are no longer symmetric because the first part
of $H_P$ is considered in both sets of variables.

Third, we require that each vertex can only be matched one a single
hyperedge and so have clauses that exclude matching hyperedges that
overlap non-trivially (on one or two coordinates).  
\begin{equation}
  \Psi_P^{\le 1} =
    \bigwedge_{i \in \set{1,2}} \bigwedge_{(u,v), (u',v') \in P^2} (u = u' \vee v = v') \wedge (u,v \neq u',v') \Rightarrow \neg M_{u,v}^i \vee \neg M_{u',v'}^i.
\end{equation}
Observe that $\Psi_P^{\le 1}$ has size $\Theta(s^3)$ compared to the
size of the original $\Phi_P^{\le 1}$ that had size $\Theta(s^5)$.
Fourth, we exclude the trivial 3D matching by requiring that at least
one of diagonal hyperedges not be used:
\begin{equation}
  \Psi_P^{\text{non-trivial}} = \bigvee_{u \in P} \neg M_{u,u}^1 \vee \neg M_{u,u}^2.
\end{equation}
Finally, we AND these into the overall formula:
\begin{equation}
  \Psi_P = \Psi_P^{\text{non-edge}} \wedge \Psi_P^{\le 1} \wedge \Psi_P^{\ge 1} \wedge \Psi_P^{\text{non-trivial}}.
\end{equation}
The size of the CNF formula $\Psi_P$ is $\Theta(s^3)$, a polynomial improvement over $\Phi_P$.  

Thus we reduce 3D matching to satisfiability by converting the
instance $H_P$ into the CNF formula $\Psi_P$.  At this point the
reduction is straightforward, but is included for completeness in
\autoref{alg:sat}.

\begin{algorithm}
  \caption{: Reduction to satisfiability}
  \label{alg:sat}
\begin{algorithmic}[1]
  \Require{An $(s,k)$-puzzle $P$.}
  \Ensure{YES, if $P$ is a strong USP and NO otherwise.}
  \Function{VerifySAT}{$P$}
  \State{Construct 3DM instance $H_P$.}
  \State{Construct CNF formula $\Psi_P$.}
  \If{$\Psi_P$ is satisfiable}
  \State{\Return NO}
  \Else
  \State{\Return YES}
  \EndIf
  \EndFunction
\end{algorithmic}
\end{algorithm}

As mentioned before, to implement Line 4 of \autoref{alg:sat} in
practice we used the satisfiability solver MapleCOMPSPS, from the 2016
international SAT competition \cite{XXX - contest paper and contestant
  paper}.  This solver is conflict driven and uses a learning rate
branching heuristic to decide which variables are likely to lead to a
conflict.  The solver provides no polynomial-time worse run time
guarantees, but has had demonstratable success in practice.

%% \begin{algorithm}
%%   \caption{: Create CNF formula in a Solver (Part 1)}
%%   \label{alg:cnf}
%% \begin{algorithmic}[1]
%%   \Require{An $(s,k)$-puzzle $P$ and a Solver $S$}
%%   \Ensure{Solver $S$ with CNF clauses loaded.}
%%   \Function{ConstructReduction}{$P, S$}
  
%%   \State{Construct $H_p$}
%%   \State{Allocate $2s^2$ variables space in Solver $S$}
%%   \State{Create an empty clause $C$}
%%   \State{\Comment Adding constraints (variables) from $H_P$}
%%   \For{$r_1 = 0$ to $s$}             
%%     \For{$r_2 = 0$ to $s$}
%%       \State{Get constriant $Var_{r_2}$ from $Var$ with index from $X'$ according to $r_1, r_2$ }
%%       \For{$r_3 = 0$  to $s$}
%% 	\State{Get constriant $Var_{r_3}$ from $Var$ with index from $Y'$ according to $r_1, r_3$}
%%         \If {$(r_1, r_2, r_3)$ not in $H_p$}
%% 	  \State{Clear clause $C$}
%% 	  \State{Push $Var_{r_2}$ and $Var_{r_3}$ into $C$}
%% 	  \State{Add clause $C$ into Solver $S$}
%%         \EndIf
%%       \EndFor
%%     \EndFor
%%   \EndFor
  
%%   \State{\Comment Checking the uniqueness of the edges in the 3D matching}
%%   \State{\Comment Within layer $r_1$}
%%   \State{\Comment For all $r_1$, $r_{21} != r_{22}$, and $r_{31} != r_{32}$ }
%%   \For{$r_1 = 0$ to $s$}
%%       \For{$a = 0$ to $s$}
%%       \State{Get constriant $Var_{r_{21}}$ from $Var$ with index from $X'$ according to $a, r_1$}
%%       \State{Get constriant $Var_{r_{31}}$ from $Var$ with index from $Y'$ according to $a, r_1$}
%%       	\For{$b = 0$ to $s$}
%%         \State{Get constriant $Var_{r_{22}}$ from $Var$ with index from $X'$ according to $b, r_1$}
%%         \State{Get constriant $Var_{r_{32}}$ from $Var$ with index from $Y'$ according to $b, r_1$}
%%         \If {$r_{21} < r_{22}$}  \Comment implies $r_{31} < r_{32}$
%% 	  \State{Clear clause $C$}
%% 	  \State{Push $Var_{r_{21}}$ and $Var_{r_{22}}$ into $C$}
%% 	  \State{Add clause $C$ into Solver $S$}
%% 	  \State{Clear clause $C$}
%% 	  \State{Push $Var_{r_{31}}$ and $Var_{r_{32}}$ into $C$}
%% 	  \State{Add clause $C$ into Solver $S$}
%%         \EndIf
%%         \EndFor
%%       \EndFor
%%   \EndFor
  
%%   \State{\Comment Checking the uniqueness of the edges in the 3D matching}
%%   \State{\Comment Within slice $r$}
%%   \State{\Comment For all $r_2, r_3$, $r_{11} != r_{12}$ }
%%   \For{$r = 0$ to $s$}
%%       \For{$r_{11} = 0$ to $s$}
%%       \State{Get constriant $Var_{r_{21}}$ from $Var$ with index from $X'$ according to $r_{11},r$}
%%       \State{Get constriant $Var_{r_{31}}$ from $Var$ with index from $Y'$ according to $r_{11},r$}
%%       	\For{$r_{12} = 0$ to $s$}
%%         \State{Get constriant $Var_{r_{22}}$ from $Var$ with index from $X'$ according to $r_{12}, r$}
%%         \State{Get contsriant $Var_{r_{32}}$ from $Var$ with index from $Y'$ according to $r_{12}, r$}
%%         \If {$r_{21} < r_{22}$}  \Comment implies $r_{31} < r_{32}$
%% 	  \State{Clear clause $C$}
%% 	  \State{Push $Var_{r_{21}}$ and $Var_{r_{22}}$ into $C$}
%% 	  \State{Add clause $C$ into Solver $S$}
%% 	  \State{Clear clause $C$}
%% 	  \State{Push $Var_{r_{31}}$ and $Var_{r_{32}}$ into $C$}
%% 	  \State{Add clause $C$ into Solver $S$}
%%         \EndIf
%%         \EndFor
%%       \EndFor
%%   \EndFor
%%   \algstore{bkbreak}
%%   %\EndFunction
%% \end{algorithmic}
%% \end{algorithm}

%% \addtocounter{algorithm}{-1}
%% \begin{algorithm}[h]
%%   \caption{: Create CNF formula in a Solver (Part 2)}
%% \begin{algorithmic}[1]
%% \algrestore{bkbreak}

%%   %\Comment Checking Existence of edges in layer $r_1$
%%   \For{$r_1 = 0$ to $s$} \Comment Checking Existence of edges in layer $r_1$
%%     \State{Clear clause $C$}
%%     \For{$r_2 = 0$ to $s$}
%%         \State{Get constriant $Var_{r_{2}}$ from $Var$ with index from $X'$ according to $r_{1},r_2$}
%%         \State{Push $Var_{r_{2}}$ into $C$}
%%     \EndFor
%%     \State{Add clause $C$ into Solver $S$}
    
%%     \State{Clear clause $C$}
%%     \For{$r_3 = 0$ to $s$}
%%         \State{Get constriant $Var_{r_{3}}$ from $Var$ with index from $Y'$ according to $r_{1},r_3$}
%%         \State{Push $Var_{r_{3}}$ into $C$}
%%     \EndFor
%%     \State{Add clause $C$ into Solver $S$}
%%   \EndFor
%%   \For{$r = 0$ to $s$} \Comment Checking Existence of edges in slice $r$
%%     \State{Clear clause $C$}
%%     \For{$r_1 = 0$ to $s$}
%%         \State{Get constriant $Var_{r_{x}}$ from $Var$ with index from $X'$ according to $r_{1},r$}
%%         \State{Push $Var_{r_{x}}$ into $C$}
%%     \EndFor
%%     \State{Add clause $C$ into Solver $S$}
    
%%     \State{Clear clause $C$}
%%     \For{$r_1 = 0$ to $s$}
%%         \State{Get constriant $Var_{r_{y}}$ from $Var$ with index from $Y'$ according to $r_{1},r$}
%%         \State{Push $Var_{r_{y}}$ into $C$}
%%     \EndFor
%%     \State{Add clause $C$ into Solver $S$}
%%   \EndFor
%%   \State{Clear clause $C$}
%%   \For{$r_1 = 0$ to $s$} \Comment check non-trivial cases
%%   	\State{Get constriant $Var_{rxx}$ from $Var$ with index from $X'$ according to $r_{1},r_1$}
%% 	\State{Get constriant $Var_{ryy}$ from $Var$ with index from $Y'$ according to $r_{1},r_1$}
%%   \EndFor
%%   \State{Add clause $C$ into Solver $S$}
%%   \State{\Return{ Solver $S$}}
%%   \EndFunction


%% \end{algorithmic}
%% \end{algorithm}

 
\subsection{3DM to MIP}

Another way to utilize the connection between verification of strong
USP and 3D matching is to further reduce 3D matching to integer
programming, another well-known \NP{}-complete optimization problem.
Let $H_P = \langle V, E \rangle$ be the 3D matching instance
associated with the puzzle $P$.  Our goal is to construct a
non-trivial 3D matching $M$ that is a subset of the edges of $E$ and
is vertex disjoint.  We use $M_{u,v,w}$ to denote a variable with
values in $\set{0,1}$ that indicates whether the edge $(u,v,w) \in P^3$
is present in the matching.

To ensure that $M$ is a subset of $E$ we require the following edge constraints
\begin{equation}
  \forall u,v,w \in P,~~M_{u,v,w} \le \begin{cases} 0, (u,v,w) \not\in E, and \\ 1, (u,v,w) \in E.\end{cases} \label{eqn:cons1}
\end{equation} We also require that each vertex in
each of the three parts of the graph is incident to exactly one edge
in $M$.  This is captured by the following vertex constraints:
\begin{equation}
  \label{eqn:cons2}
\begin{aligned}
&\forall w \in P,~~\sum_{u,v \in P} M_{u,v,w} = 1,\\
&\forall v \in P,~~\sum_{u,w \in P} M_{u,v,w} = 1,\text{and} \\
&\forall u \in P,~~\sum_{v,w \in P} M_{u,v,w} = 1.
\end{aligned}
\end{equation}
Finally, since we need that the 3D matching be non-trivial we
add the constraint
\begin{equation}
  \label{eqn:cons3}
  \sum_{u \in P} M_{u,u,u} < |P|.
\end{equation}
With these constraints in mind, \autoref{alg:mip} describes the
overall reduction. 

%% By this theory, \autoref{alg:mip} would first transfers input puzzle
%% $P$ into a tripartite hypergraph $H_p$ and creates an empty mixed
%% integer programming model. For every hyperedge $e= \notin H_p$, a
%% constrain $f(e) = 0$ is added into the model as it can not be selected
%% into subset $M$. Afterward the algorithm would also add the
%% restrictions listed in the paragraph above (I am not sure if I should
%% list all the constrains again?). Finally, the model is put through a
%% integer programming solver. If the model has a solution, $H_p$ has a
%% nontrivial 3D matching and by \autoref{lem:verify-to-3dm} $P$ is not a
%% USP. Otherwise if the model is infeasible, $H_p$ does not have a
%% nontrivial thus $P$ is a USP.

\begin{algorithm}
  \caption{: Reduction to mixed integer programming}
  \label{alg:mip}
\begin{algorithmic}[1]
  \Require{An $(s,k)$-puzzle $P$.}
  \Ensure{YES, if $P$ is a strong USP and NO otherwise.}
  \Function{VerifyMIP}{$P$}
  \State{Construct 3DM instance $H_P$.}
  \State{Create empty integer program $Q$ with Boolean variables $\condset{M_{u,v,w}}{u,v,w \in P}$.}
  \State{Add constraints of \autoref{eqn:cons1}, \autoref{eqn:cons2}, and \autoref{eqn:cons3} to $Q$.}
  \If{\textsc{IsFeasible}$(Q)$}
  \State{\Return{NO}.}
  \Else
  \State{\Return{YES}.}
  \EndIf
  \EndFunction
\end{algorithmic}
\end{algorithm}

The heavy lifting in \autoref{alg:mip} occurs in Line 5 where the
integer program is checked for feasibility, i.e., that there is an
assignment to the variables $M$ that satisfy all the constraints.  In
practice we hand off this computation to the third-party mixed-integer
programming solver Gurobi \cite{gurobi}.  It should be noted that
Gurobi is a commerical, closed-source solver, so we cannot necessarily
verify its assertion that the integer program $Q$ is infeasible.  On
the otherhand, when it determines that the integer program $Q$ is
feasible, we can verify that the solution it produced is a witness of
feasibility.  Further, Gurobi provides no asymptotic performance
guarantees, must less polynomial ones.  We do note that reduction from
3D matching to IP is polynomial time as $H_P$ can be efficiently
constructed from $P$ and there are $s^3 + 3s^3 + 1 = O(s^3)$
constraints and $s^3$ variables in the integer program $Q$.
 
\label{subsec:mip}




\subsection{Heuristics}
\label{sec:heuristic}

Despite our efforts devise an exact algorithm for verifying strong
USPs, the performance of the final algorithm was not practical.  To
speed the algorithm up we considered a number of verification
heuristics that were fast to evaluate, but did not alway produce a
definitive answer.  The heuristics output YES, NO, or MAYBE to whether
a given puzzle $P$ was a strong USP.  We consider only Las Vegas
algorithms here, that is, algorithms whose output of YES or NO is
always correct.  Further, all the heuristics we consider are one-sided
in that they for all input they either return YES or MAYBE, or NO or
MAYBE.  To verify a puzzle $P$ we run a battery of fast heuristics and
return early if any of the heuristics produce a definitive YES or NO.
When all the heuristics result in MAYBE then run one of the slower
exact algorithms that were discussed in the previous section.  Most of
the heuristics we considered were deterministic, but a few were
randomized.  This is again in the Las Vegas sense: An output of YES or
NO is always correct, but as function of the input and internal
algorthmic randomness MAYBE can be also be output and can vary between
independent runs.  The heuristics have several different forms, but
all rely ultimately on structural properties of strong USP.

\subsubsection{Downward Closed}

The simplest heuristics we considered were based on the fact that
strong USP are downward closed, that is, if $P$ is a strong USP, then
so is every subpuzzle $P' \sse P$.  This leads to a practical heuristic
that can determine that a puzzle is not a strong USP.

\begin{algorithm}
  \caption{: Downward-closed Heuristic}
  \label{alg:downward-closed}
\begin{algorithmic}[1]
  \Require{An $(s,k)$-puzzle $P$, and size $s' \le s$.}
  \Ensure{NO, if $P$ has a set of $s'$ rows that do not form a strong USP, and MAYBE otherwise.}
  \Function{HeuristicDownwardClosed}{$P, s'$}
  \For{$P' \sse P, |P'| = s'$}
      \If{$P'$ is not a strong USP} \Return{NO.} \EndIf
  \EndFor{}
  \State{\Return{MAYBE}.}
  \EndFunction
\end{algorithmic}
\end{algorithm}

This algorithm runs in time $O(\binom{s}{s'} T_{Verify}(s', k))$ where
$T_{Verify}(s',k)$ is the running time for exactly verifying a
$(s',k)$-puzzle.  In practice we did not apply this heuristic for $s'$
larger than $3$, so the effective running time was $O(s^3 T(3,k))$,
which is polynomial in $s$ and $k$ using the verification algorithms
from the previous subsections that eliminate dependence on $k$ for
polynomial cost.  This heuristic can be made even more practical by
caching the results for puzzles of size $s'$, reducing the
verification time per iteration to constant in exchange for
$O(\binom{3^k}{s'}T(s',k))$ time to precompute the values for all puzzles of
size $s'$.  There is also space overhead because the precomputed
results are being cached.  We also note that for a puzzle $P$ that is
a strong USP, the heuristic takes $\Theta(s^3 T(3,k))$ time without
cache or $\Theta(s^3)$ with caching.

Note that since our search algorithms typically start from a known
strong USP and tries to add rows, looking a subsets of a puzzle that
have already been verified is wasteful, and can be skipped over in
this setting.  From a practical point of view, running this heuristic
is free for small constant $s'$, as the exact verification algorithms
have a matching or higher polynomial running time.  Furthermore, since
the algorithm can return early, its expected running time on random
non-strong USPs is low.  There appeared to be diminishing returns with
increasing $s'$ as it substantially increases precompute time and
storage while each stored value contributes relatively less benefit to
verification as there are now many more values stored and are individually less likely to be examined.

%% \subsubsection{Random}

%% XXX - There's also a random reordering heuristic, but I don't remember
%% the justification for it.  It's more elaborate than the other, but
%% less motivated.

%% \begin{algorithm}
%%   \caption{: Random Heuristic}
%%   \label{alg:random}
%% \begin{algorithmic}[1]
%%   \Require{An $(s,k)$-puzzle $P$, and iteration bound $t$.}
%%   \Ensure{NO, if a witness is found for $P$ not being a strong USP, and MAYBE otherwise.}
%%   \Function{HeuristicRandom}{$P$}
%%   \State{XXX - Fill out.}
%%   \EndFunction
%% \end{algorithmic}
%% \end{algorithm}

\subsubsection{Greedy}

This heuristic attempts to solve the 3D-Matching instance associated
with verifying a puzzle $P$ (discussed in \autoref{sec:3DM}).  The
heuristic proceeds iteratively, determining the vertex (of the first part) of the 3DM
instance with the least edges and randomly selecting a hyperedge of that vertex to put into the 3DM.  If the heuristic successfully contructs a
3DM it returns NO indicating that the input puzzle $P$ is not a strong
USP.  If the heuristic reaches point were prior commits have made the
matching infeasible, the heuristic starts again from scratch.  This
process is repeated some number of times before it gives up and
returns MAYBE.  In our practical implementation we use $s^3$
iterations because the time balances well with the other heuristics
and empricially reduced the number of instances requiring a full
verification.

\begin{algorithm}
  \caption{: (Random) Greedy Heuristic}
  \label{alg:random-greedy}
\begin{algorithmic}[1]
  \Require{An $(s,k)$-puzzle $P$, and iteration bound $t$.}
  \Ensure{NO, if a witness is found for $P$ not being a strong USP, and MAYBE otherwise.}
  \Function{HeuristicGreedy}{$P$}
  \State{Construct 3DM instance $H_P$.}
  \For{$i = 1$ to $t$}
    \For{$u \in P$}
      \State{$cts[u] = \sum_{v,w \in P} H_P(u,v,w)$.} \Comment{Number of edges incident vertex $u$.}
    \EndFor
    \State{Let $U,V,W = \emptyset.$}
    \State{Let $m = 0.$} \Comment{Number of hyperedges in matching.}
    \While{$m < s$} 
      \State{Select $u \in \condset{u \in \bar{U}}{cts[u] = \max_{v \in \bar{U}} cts[v]}$ uniformly at random.}
      \If{$cts[u] = 0$} break. \EndIf
      \State{Select $(v,w) \in \condset{(v,w) \in \bar{V} \times \bar{W}}{H_P(u,v,w) = 1}$ uniformly at random.}
      \For{$v' \in P$} \Comment{Update edge counts.}
        \For{$w' \in P$}
          \If{$(v',w') \in \bar{V} \times \bar{W}$ and $H_P(u,v',w') = 1$} $cts[u]\texttt{--}$. \EndIf
          \If{$(v',w') \in \bar{U} \times \bar{W}$ and $H_P(v',v,w') = 1$ and $v' \neq u$} $cts[v']\texttt{--}$. \EndIf
          \If{$(v',w') \in \bar{U} \times \bar{V}$ and $H_P(v',w',w) = 1$ and $v' \neq u$ and $v' \neq v$} $cts[v']\texttt{--}$. \EndIf
        \EndFor
      \EndFor
      \State{$U = U \cup \set{u}$.} \Comment{Add edge to matching.}
      \State{$V = V \cup \set{v}$.}
      \State{$W = W \cup \set{w}$.}
      \State{$m = m + 1.$}
      \EndWhile
    \If {$m > s$} \Return{NO}. \Comment{3DM found so not SUSP, halt.} \EndIf
  \EndFor
  \State{\Return{MAYBE}.}
  \EndFunction
\end{algorithmic}
\end{algorithm}

The array $cts$ is used to store the number of edges $cts[u]$
that remain associated with vertex $u$ along the first coordinate.
Much of the algorithm is devoted to maintaining this invariant.  The
sets $U,V,W$ store the vertices along the three coordinates,
respectively, that have already been incorporated into the partial 3D
matching.  Like in \autoref{alg:bi} we do not store the
matching itself, only the vertices involved.  The break at Line 10
triggers when the partial 3D matching is a dead end and cannot be
extended into a full 3D matching.  The condition of Line 21 is true
when a full 3D matching has been constructed and causes the algorithm
to return that $P$ is not a strong USP.

The running time of this algorithm is $O(s^3 t + T_{3DM}(s,k))$, where
$T_{3DM}(s,k)$ is the time required to construct 3D matching
instances from $(s,k)$-puzzles.  This algorithm has the potential to
be considerably slower than the downward-closure heuristic, and in practice we set $t = s^3$.  However, the main loop can terminate early at Line 10
when it fails to extend the 3D matching this permits the expected time
to much less than the worst case.  For a puzzle $P$ that is a strong
USP, the heuristic takes the full $\Omega(s^3 t + T_{3DM}(s,k))$ time.


\begin{comment}


\subsubsection{Graph Automorphism}

A strong uniquely-solvable puzzle must also be a uniquely-solvable
puzzle.  Given a puzzle $P$ we can construct a graph $G_P$ such that
$G_P$ is rigid iff $P$ is a uniquely-solvable puzzle.

XXX - I don't think this idea worked out.  The code correctly
implemented the approach, but the approach was flawed.  Probably
remove this section or add to future work.

XXXX - I don't remember the argument either way...

\end{comment}

\subsubsection{2D Matching}

The final heuristic we present is one-sided in the opposite direction
of the others, it may return YES or MAYBE.  In order for a hypergraph
$H_P$ to have a 3D
matching it must be the case that when any one of the three parts of
vertices of $H_P$ is projected away the resulting bipartite graph has a
perfect matching.  Thus we can witness that there is no 3D matching by
determining that one of three projected bipartite graphs $H_P|_1, H_P|_2,
H_P|_3$ does not have a perfect matching.

\begin{algorithm}
  \caption{: 2D Matching Heuristic}
  \label{alg:2dm}
\begin{algorithmic}[1]
  \Require{An $(s,k)$-puzzle $P$.}
  \Ensure{YES, if $P$ is found to be strong USP, and MAYBE otherwise.}
  \Function{Heuristic2DMatching}{$P$}
  \State{Construct 3DM instance $H_P.$}
  \State{\texttt{// Construct projections.}}
  \State{Define projection $H_P|_1(b,c) = \vee_{a \in [s]} H_P(a,b,c)$.}
  \State{Define projection $H_P|_2(a,c) = \vee_{b \in [s]} H_P(a,b,c)$.}
  \State{Define projection $H_P|_3(a,b) = \vee_{c \in [s]} H_P(a,b,c)$.}

  \If{$H_P|_1, H_P|_2,$ and $H_P|_3$ have bipartite perfect matchings}
  \State{\Return{MAYBE.}}
  \Else
  \State{\Return{YES.}}
  \EndIf
  \EndFunction
\end{algorithmic}
\end{algorithm}

We can efficiently decide bipartite perfect matching using the
standard reduction to network flow and solve it using the
Ford-Faulkerson augmenting flow algorithm.  This approach yields an
algorithm that runs in time $O(s^3 + T_{3DM}(s,k))$, where
$T_{3DM}(s,k)$ is the time required to construct 3D matching instances
from $(s,k)$-puzzles.

In practice we found this heuristic not to be effective because the
projected $H_P|_i$ are typically dense graphs even if $H_P$ is not, and so
they are likely to have perfect matchings independently of whether $H_P$
has a 3D matching.



\subsection{Hybrid Algorithm}

Our final verification algorithm (\autoref{alg:hybrid}) is a hybrid of several verification
algorithms and heuristics.  The size thresholds for which algorithm
and heuristic to apply were determined experimentally for small $k$
and were focused on the values were our strong USP search algorithms
were tractable $k \le 6$ (or nearly tractable $k \le 8$).  We decided
to run both the reduction to SAT and MIP in parallel because it was
not clear which algorithm performed better.  Since verification halts
when either algorithm completes the wasted effort is within a factor
of two of what the better algorithm could have done alone.  We also
chose to do this because we experimentally observed that there were
many instances that one of the algorithms struggled with that the
other did not.  We end up not using \textsc{Heuristic2DMatching}
because it rarely produced definitive output.

\begin{algorithm}
  \caption{: Hybrid Verification Algorithm}
  \label{alg:hybrid}
\begin{algorithmic}[1]
  \Require{An $(s,k)$-puzzle $P$.}
  \Ensure{YES, if $P$ is found to be strong USP, and MAYBE otherwise.}
  \Function{Verify}{$P$}
  \If{$s \le 2$} \Return{\Call{VerifyBruteForce}{$P$}.} \EndIf
  \If{$s \le 7$} \Return{\Call{VerifyBidirectional}{$P$}.} \EndIf
  \If{$s \le 10$}
    \State{Return result if \Call{HeuristicDownwardClosed}{$P, 2$} is not MAYBE.}
    \State{\Return{\Call{VerifyBidirectional}{$P$}.}}
    \EndIf
  \State{Return result if \Call{HeuristicDownwardClosed}{$P, 3$} is not MAYBE.}
  \State{Return result if \Call{HeuristicRandom}{$P$} is not MAYBE.}
  \State{Return result if \Call{HeuristicGreedy}{$P$} is not MAYBE.}
  \State{Run \Call{VerifySAT}{$P$} and \Call{VerifyMIP}{$P$} in parallel and return the first result.}
  \EndFunction
\end{algorithmic}
\end{algorithm}

\section{Searching for Strong USPs}
\label{sec:search}

In certain respects, the problem of constructing a large strong USP is
much like the problem of constructing a large set of linearly
independent vectors in a vector space.  Indeed, the object to be
constructed is a set and the order that elements are added does not
matter, further the underlying elements are represented as a vector.
There are polynomial-time algorithms for determining whether a set of
vectors are independent, and we have a practical algorithm for
deciding whether a puzzle is a strong USP.

There is a straightforward algorithm for constructing maxmimum size
sets of independent vectors: Start with an empty set $S$, and
repeatedly add vectors to $S$ which are linearly independent of the
vectors currently in $S$.  After this process completes $S$ is a
largest set of linearly independent vectors.  This problem admits such
a greedy algorithm because the family of sets of linearly independent vectors
form a matroid.  The vector to be added each step can be computed
efficiently by solving a linear system of equations for vectors in the
null space of $S$.  Altogether this gives an efficient algorithm to
compute a maximum set of independent vectors or, more generally, to
complete a maximum set with some given subset $S'$.

Unfortunately this same approach does not work for generating maximum
size strong USPs.  The set of strong USP do not form a matroid, rather
they only form an independence system \cite{XXX}.  In particular, the
empty puzzle is a strong USP and the set of strong USPs are downward
closed, so that if $P_2$ is a strong USP and $P_1 \subseteq P_2$, then
$P_1$ is a strong USP.  The third and final property required to be a
matroid, the augmentation property, requires that for every pair of
strong USPs $P_1, P_2$ with $|P_1| \le |P_2|$ there is a row of $r \in
P_2 \backslash P_1$ such that $P_1 \cup \set{r}$ is also a strong USP.
A simple counterexample with the strong USPs $P_1 = \set{32}$ and $P_2
= \set{12, 23}$ concludes that neither $P_1 \cup \set{12} = \set{12,
  32}$ nor $P_1 \cup \set{23} = \set{23, 32}$ are strong USP, and
hence the augmentation property does not hold.

One consequence is that wholly greedy algorithms will not be effective
for finding maximum size strong USPs \cite{XXX}.  Furthermore, we do not
currently have a practical or efficient algorithm that can take a
strong USP $P$ and determine a single row, or set of rows, that can
extend $P$ while maintaining that it is a strong USP.

That said, we have had some success applying general purpose search
techniques together with our practical verification algorithm to
construct maximum size strong USPs for small $k$.  In particular, we
implemented variants of depth-first search (DFS) and breadth-first
search (BFS) in both desktop and HPC settings.

A width-$k$ puzzle has potentially $3^k$ distinct rows.  For $j \in
[3^k]$ we use $r_j$ to denote the $j^{th}$ row in lexicographic order.
For a puzzle $P$ and integer $i \in [|P|]$ we use $P_i$ to denote the
$i^{th}$ row of $P$ where the rows are sequenced in lexicographic
order. We include pseudocode of both searches for completeness.

\begin{algorithm}
  \caption{: Depth-First Search}
  \label{alg:dfs}
\begin{algorithmic}[1]
  \Require{An integer $k \ge 0$, and a width-$k$ strong USP $P$.}
  \Ensure{The number $best$, which is the size of the largest strong USP that has $P$ as a subpuzzle.}

  \Function{DFS}{$P$}

  \State{Let $best = |P|$.}
  \State{Let $j$ be the index of the row $P_{|P|-1}$ in lexicographic order.}

  \For{$r = r_{j+1}$ to $r_{3^k-1}$}
    \State{Let $P' = P \cup \set{r}$.}
    \If{\Call{Verify}{$P'$}}
      \State{$best = \max(best, $ \Call{DFS}{$k, P'$}).}
    \EndIf
  \EndFor

  \State{\Return{$best$}.}

  \EndFunction
\end{algorithmic}
\end{algorithm}

\begin{algorithm}
  \caption{: Breadth-First Search}
  \label{alg:bfs}
\begin{algorithmic}[1]
  \Require{An integer $k \ge 0$.}
  \Ensure{The number $best$, which is the size of the largest strong USP that has $P$ as a subpuzzle.}

  \Function{BFS}{$k$}

  \State{Let $Q$ be an empty queue.}
  \State{Push $\emptyset$ onto end of $Q$.}

  \While{$Q$ is not empty}
    \State{Pop head of $Q$ into $P$.}
    \State{Let $j$ be the index of the row $P_{|P|-1}$ in the lexicographic order of all rows.}
    \For{$r = r_{j+1}$ to $r_{3^k-1}$}
      \State{Let $P' = P \cup \set{r}$.}
      \If{\Call{Verify}{$P'$}}
        \State{Push $P'$ onto end of $Q$.}
        \State{$best = |P'|$.}
      \EndIf
    \EndFor
  \EndWhile

  \State{\Return{$best$}.}

  \EndFunction
\end{algorithmic}
\end{algorithm}

Arguing the correctness of these algorithms is routine.  It should be
noted that since we treat the puzzles as though they have rows sorted
in lexicographic order, we only need to consider adding rows that
occur later in the order than the highest row already in the puzzle,
which is $r_j$.  This explains the initial row $r_{j+1}$ being used in
the for loop in both algorithms.  Observe that \Call{BFS}{$k$} $=$
\Call{DFS}{$k, \emptyset$} is the maximum size of a strong USP of
width $k$.  The worst case running time of these algorithms is
similarly routine, they both run in time $O(3^k \cdot \#USP(k) \cdot
T_{\textsc{Verify}}($\Call{BFS}{$k}$, $k))$ where $\#USP(k)$ is the
number of strong USPs of width $k$ and
$T_{\textsc{Verify}}(BFS(k)+1,k)$ is the time to verify the maximum
size puzzle examined by the algorithm (assuming $T_{\textsc{Verify}}$
is monotone in its parameters).

The actual running times of both these algorithms are prohibitive even
for $k = 5$, and the greater memory usage of \textsc{BFS} to store
the entire search frontier in $Q$ is in the terabytes for $k = 6$.  There
are some silver linings, \textsc{DFS} can report intermediate results
that are the maximal strong USPs that it has discovered so far.  Both
algorithms admit the possibility of eliminating puzzles from the
search that are equivalent to puzzles that have already been searched,
though it is easier to fit into the structure \textsc{BFS} as the
puzzles are already being stored in a queue.

\begin{comment}
  Not planning to discuss:
  \begin{itemize}
  \item A$^*$ + admissible heuristics.
  \item Upper bounds from A$^*$.
  \item Symmetry removal.
  \end{itemize}
\end{comment}

\section{Experimental Results}
\label{sec:results}

Our experimental results come in three flavors for small constant
values of $k$: (i) constructive lower bounds on the maximum size of
width-$k$ strong USP, (ii) exhaustive, non constructive, upper
bounds on the maximum size of width-$k$ strong USP, and (iii)
experimental run times comparing the various algorithms for verifying
width-$k$ strong USP.  Here constructive means that specific concrete
strong USPs are produced by our implementation that witness the bound.
\textsc{BFS} and \textsc{DFS} when able to run to search the entire space to completion provide
results of form (i) and (ii) and are tight.  When unable to run to
completion \textsc{DFS} provides only result of form (i), and, as
such, are not guaranteed to be tight.

\subsection{New Bounds on the Size of Strong USPs}
\label{subsec:usps_found}

We begin with a diagram, \autoref{fig:examples}, of representative
examples of maximal strong USPs we found for $k \le 6$.  For space
considerations, we omit larger examples found from this article,
though they are included with the source code of this project.

\begin{figure}
  \label{fig:examples}
  \begin{multicols}{4}
  (1,1):\\[.5ex]
  \begin{tabular}{|c|}
    \hline
    1 \\ \hline
  \end{tabular}\\[6ex]

  (2,2):\\[.5ex]
  \begin{tabular}{|c|c|}
    \hline
    1&3 \\ \hline
    2&1 \\ \hline
  \end{tabular}\\[16ex]

  (3,3):\\[.5ex]
  \begin{tabular}{|c|c|c|}
    \hline
    1&1&1 \\ \hline
    3&2&1 \\ \hline
    3&3&2 \\ \hline
    \end{tabular}\\[2ex]

  (5,4):\\[.5ex]
  \begin{tabular}{|c|c|c|c|}
    \hline
    3&1&3&2 \\ \hline
    1&2&3&2 \\ \hline
    1&1&1&3 \\ \hline
    3&2&1&3 \\ \hline
    3&3&2&3 \\ \hline
  \end{tabular}\\[10ex]

  (8,5):\\[.5ex]
  \begin{tabular}{|c|c|c|c|c|}
    \hline
    3&3&3&1&1 \\ \hline
    1&1&2&2&1 \\ \hline
    2&1&3&3&2 \\ \hline
    3&2&2&2&3 \\ \hline
    2&1&2&1&3 \\ \hline
    2&2&3&1&2 \\ \hline
    3&2&3&2&1 \\ \hline
    3&1&2&1&1 \\ \hline
  \end{tabular}\\[16ex]

  (14,6):\\[.5ex]
  \begin{tabular}{|c|c|c|c|c|c|}
    \hline
    2&3&3&1&1&1 \\ \hline
    2&1&1&2&1&1 \\ \hline
    3&3&1&2&1&1 \\\hline
    3&2&2&2&1&1 \\\hline
    2&3&1&1&2&1 \\\hline
    2&2&3&1&2&1 \\\hline
    3&3&1&3&2&1 \\\hline
    3&2&3&3&2&1 \\\hline
    2&1&1&3&1&2 \\\hline
    2&3&1&3&2&2 \\\hline
    3&1&1&1&1&3 \\\hline
    3&3&2&3&1&3 \\\hline
    3&3&2&1&2&3 \\\hline
    2&2&3&2&2&3 \\\hline
  \end{tabular}
  \end{multicols}
  \caption{Respresentative examples of the largest strong uniquely
    solvable $(s,k)$-puzzles found from width $k = 1$ to $6$.}
\end{figure}

\autoref{table:compare} summarizes our main results which are improved
bounds on the maximum size of strong uniquely solvable puzzles for
small widths $k$, and compares them to bounds from \cite{cksu05}.  The
lower bounds of \cite{cksu05} are from constructions in their
Propositions 3.1 and 3.8, which give families for $k$ even or $k$
divisible by three.  Note that because the SUSP property is preserved
when extending with an arbitrary column, the lines that are missing
lower bounds could be continued with the lower bound from the next
previous line.  The upper bounds of \cite{cksu05} follow from their
Lemma 3.2 and the fact that the capacity of SUSP is bounded above by
the capacity of USP.  Observe that our experiment derives tight bounds
for all $k \le 5$ and improves the known lower bounds for $4 \le k \le 12$.
\begin{table}
  \label{table:compare}
  \begin{center}
  \begin{tabular}{|c|r|r|r|r|}
    \hline
    & \multicolumn{2}{|c|}{[CKSU05]} & \multicolumn{2}{|c|}{This work} \\
    \hline
    $k$ & Maximum $s$ & $\omega^*$~ & Maximum $s$ & $\omega^*$~\\
    \hline
    1 & $1 = s \;\,~~~~~~~~~~~~$ & $3.000$ & $1=s$ & $3.000$  \\
    2 & $2 \le s\le ~~~~~~3$ & $2.671$ & $2=s$ & $2.671$ \\
    3 & $3 \le s \le ~~~~~~6$ & $2.643$ & $3=s$ & $2.643$ \\
    4 & $4 \le s\le ~~~~12$ & $2.671$ & $5=s$ & $2.586$ \\
    5 & $ s\le ~~~~24$ &  & $8=s$ & $2.563$  \\
    6 & $10 \le s \le ~~~~45$ & $2.616$ &$14\le s$ & $2.522$\\
    7 & $s\le ~~~~86$ &  & $21\le s$ & $2.531$ \\
    8 & $16 \le s\le ~~162$ & $2.671$ & $30\le s$ & $2.548$ \\
    9 & $36 \le s \le ~~307$ & $2.593$ &$42\le s$ & $2.564$  \\
    10 & $s \le ~~581$ & & $64 \le s$ & $2.563$ \\ 
    11 & $s \le 1098$ &  & $112 \le s$ & $2.541$ \\
    12 & $136 \le s \le 2075$ & $2.574$ & $196 \le s$ & $2.522$ \\
    \hline
  \end{tabular}
  \end{center}
  \caption{Comparison of bounds on maximum size of strong uniquely
    solvable $(s,k)$-puzzles with \cite{cksu05}.  $\omega^*$ is the
    $\omega$ in the limit of composing puzzles of these dimensions via
    direct product.}
\end{table}
We conjecture that $s = 14$ is tight for $k = 6$, given the amount of
CPU cycles and different algorithmic approaches we have applied to
this case where it quickly, on order of an hour of CPU time, gets to
$14$, and makes no further progress in the succeeding month.  We do
not have any strong beliefs about the lower bounds for larger values
of $k$ except that they are likely not tight.  Furthermore, it is
appearant that exhaustive search will not be practical for $k \ge 6$
unless the search space can be substantially reduced.

The column labelled $\omega^*$ in \autoref{table:compare} denotes the
upper bound on $\omega$ the strong USP imply when taking the direct
product of the puzzle with itself finitely mainly times, and then
applying \autoref{lem:omega}.  (XXX - explain why $\omega ^*$.)  The
lines with omitted lower bounds for \cite{cksu05} also omit
$\omega^*$, because it is strictly worst than the $\omega^*$ in
preceding lines.  Notice that the minimum $\omega^*$ occurs in the $k
= 6$ line of our work, this is where we have generated the relatively
largest size puzzle for its width.  In the subsequent lines the
$\omega^*$ is no smaller, because the SUSPs found are relatively
smaller in comparison to their width.  Note that $\omega^*$ for $k =
12$ matches that of $k = 6$.  This is because the $(196, 12)$-puzzle
is the result of taking the Cartesian product of the $(14,6)$-SUSP
with itself, and hence is the same by definition of $\omega^*$.

The most interesting line of \autoref{table:compare} is $k = 6$.  Here
we have the greatest improvement in $\omega^*$ versus \cite{cksu05}.
Ultimately, the construction of \cite[Proposition 3.8]{cksu05} gives
an infinite family of strong USPs of size that achieve $\omega <
2.48$. Our results suggests that there is considerable room for
improvement beyond that.


\subsection{Algorithm Performance}
\label{subsec:performance}

XXX - Tables / plots of data on verify's performance.

XXX - Tables / plots of data on search performance?

\section{Conclusions}
\label{sec:conclusion}

We initiated the first study of the verification strong USP and
developed practical software for both verifying and searching for strong
USP.  We give tight results on the maximum size of width-$k$ strong
USP for $k \le 5$.  Although our results do not produce a new upper
bound on the running time of matrix multiplication, they demonstrate
there is promise in the uniquely solvable puzzle approach.  The
immediate open question is: What are tight bounds on maximum size
strong USP for $k \ge 6$ and do these bound lead to asymptotically
faster algorithms for matrix multiplication?  The main bottleneck in
our work is the size of the search space.  This leads to a number of
open questions: Can the size of the search space be lowered by
additional reductions in symmetry?  We have some preliminary data that
suggests this is possible, but is not sufficient to make exhaustive
search width-6 strong USP feasible.  Is there a way to identify rows
that can be added to a strong USP other than examining all possible
rows and checking whether the puzzle remains a strong USP?  Are there
other search strategies that would be more effective on this search
space?  Is strong USP verification \NP-complete?

\bibliographystyle{customurlbst/alphaurlpp} \bibliography{references}

\appendix


\end{document}
